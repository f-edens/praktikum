%% LyX 2.1.2 created this file.  For more info, see http://www.lyx.org/.
%% Do not edit unless you really know what you are doing.
\documentclass[ngerman,english]{article}
\usepackage[T1]{fontenc}
\usepackage[latin9]{inputenc}
\usepackage{graphicx}

\makeatletter

%%%%%%%%%%%%%%%%%%%%%%%%%%%%%% LyX specific LaTeX commands.
%% Because html converters don't know tabularnewline
\providecommand{\tabularnewline}{\\}

\makeatother

\usepackage{babel}
\begin{document}
\selectlanguage{ngerman}%

\section{Durchf�hrung}

Im ersten Teil des Versuchs wurde die Z�hlrohrcharakteristik bestimmt,
daf�r wurde der $\beta$- Strahler ohne einen Absorber vor das Geiger-M�ller-Z�hlrohr
gestellt und die Impulsrate bei verschiedenen Spannungen gemessen,
um die gew�nschte Unsicherheit von unter 3\% zu erreichen ist es notwendig
mehr als 1111 Impulse zu messen.

\begin{table}
\protect\caption{\selectlanguage{english}%
Impulsrate in Abh�ngigkeit der Z�hlrohrspannung\selectlanguage{ngerman}%
}


\begin{tabular}{|c|c|c|c|c|}
\hline 
Spannung in V $\pm12,5V$ & Gemessene Impulse & Zeit in s $\pm1s$ & Impulsrate in 1/s & Fehler\tabularnewline
\hline 
\hline 
350 & 1112 & 101 & 11,01 & 0,35\tabularnewline
\hline 
400 & 1116 & 103 & 10,83 & 0,34\tabularnewline
\hline 
450 & 1117 & 98 & 11,40 & 0,36\tabularnewline
\hline 
500 & 1115 & 104 & 10,72 & 0,34\tabularnewline
\hline 
\end{tabular}
\end{table}


Anhand der gemessenen Werte ist sichtbar, dass die Spannung am Z�hlrohr
keinen signifikanten Einfluss auf die Impulsrate hat.

Im zweiten Teil des Versuchs wurde 100 mal in 10 Sekunden Intervallen
die Untergrundstrahlung gemessen, 

\begin{figure}
\protect\caption{Untergrundimpulse}
\includegraphics{C:/Users/hp/Desktop/Praktikum/Versuch_nr2}
\end{figure}


dieses Diagramm zeigt eindeutig �hnlichkeiten zu einer Poisson-Verteilung
auf!

Aus den gemessenen Werten l�sst sich der Mittelwert berechnen zu,
\[
\bar{N}=0,251\pm0,016\frac{Zerf\ddot{a}lle}{Sekunde}
\]


dies entspricht in guter N�herung einem Zerfall alle vier Sekunden!

Die Standartabweichung ist die Wurzel aus diesem Wert,
\[
\sqrt{\bar{N}}=0,50\pm0,13
\]


in einer weiteren Messreihe zu Untergrundimpulsen, wurde die Zeit
bestimmt in der mehr als 500 Impulse gemessen wurden. 

Das Ergebnis dieser Messreihe zeigte, dass 501 Impulse in einer Zeit
von $2070s\pm1s$ gemessen wurden. F�r die Impulsrate gilt dementsprechend
\[
\frac{N}{t}=(0,24\pm0,01)\frac{Zerf\ddot{a}lle}{Sekunde}
\]


dieses Ergebnis stimmt gut mit dem vorherigen �berein, n�mlich das
alle vier Sekunden ein Zerfall stattfindet!\selectlanguage{english}%

\end{document}
