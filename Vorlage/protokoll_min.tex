% ################################################################
% #                                                              #
% # Autor: Michael Epping                                        #
% # E-Mail: michael.epping@uni-muenster.de                       #
% # Version: 1.0                                                 #
% # Datum: Juni 2013                                             #
% # Info: Diese Version der Protokollvorlage besteht nur aus     #
% #       einer Datei. Außerdem wurde auf viele Spielereien      #
% #       verzichtet.                                            #
% # Copyright: CC0 (macht mit diesen Dateien was ihr wollt)      #
% #    https://creativecommons.org/publicdomain/zero/1.0/deed.de #
% #                                                              #
% ################################################################

% ###############
% # Allgemeines #
% ###############

% Zeilen, die mit einem Prozentzeichen beginnen sind Kommentare. 
% Alle verwendeten Funktionen sind mit solchen Kommentaren versehen, so dass man den Zweck der jeweiligen Funktion nachvollziehen kann.

% ######################################
% # Konfigurieren der Dokumentenklasse #
% ######################################

\documentclass[
    bibtotocnumbered,                                      % Listet das Literaturverzeichnis auch im Inhaltsverzeichnis auf
    a4paper,                                               % Papierformat
    oneside,                                               % Einseitig
    %twoside,                                              % Zweiseitig
    12pt,                                                  % Schriftgröße
    pagesize=auto                                          % schreibt die Papiergröße korrekt ins Ausgabedokument
]{scrartcl}
% Es gibt die Dokumenttypen scrartcle, srcbook, scrreprt und scrlettr. Diese gehören zum KOME-Skript und sollten für deutsche Texte benutzt werden.
% Für englische Texte wählt man entsprechend article, book, report und letter.
% Es ist nicht unbedingt zu empfehlen, bei einem bestehendem Dokument, die documentclass zu ändern.

% ####################
% # Pakete einbinden #
% ####################

% Pakete erweitern LaTeX um zusätzliche Funktionen. Dies ist eine Satz nützlicher Pakete.
\usepackage[utf8x]{inputenc}                               % Legt die Zeichenkodierung fest, z.B UTF8
\usepackage[T1]{fontenc}                                   % Verwendung der Zeichentabelle T1, für deutschsprachige Dokumente sinnvoll
\usepackage[ngerman,english]{babel}                        % Silbentrennung nach neuer deutscher und englischer Rechtschreibung
\usepackage{amsmath}                                       % Mathepaket
\usepackage{units}                                         % Ermöglicht die Nutzung von \unit[Zahl]{Einheit}

% #########################
% # Beginn des Dokumentes #
% #########################

\begin{document}
\selectlanguage{ngerman}                                   % Schreibsprache Deutsch

% Römische Ziffern als Seitenzahlen für Titelseite bis einschließlich dem Inhaltsverzeichnis
\setcounter{page}{1}
\pagenumbering{roman}

% ########################
% # Titelseite erstellen #
% ########################

\begin{titlepage}
    \vspace*{4cm}
    \begin{center}
        \Huge
        \textbf{Protokollvorlage}\\
        \vspace{1cm}
        \large
        Einfache Version \\
        \vspace{5cm}
        Michael Epping \\  
        \vspace{1cm}
        \normalsize      
        \textit{michael.epping@uni-muenster.de} \\ 
    \end{center}
\end{titlepage}

% ################################
% # Inhaltsverzeichnis erstellen #
% ################################

\tableofcontents
\newpage

% Zurücksetzen der Seitenzahlen auf arabische Ziffern
\setcounter{page}{1}
\pagenumbering{arabic}

% ###################################
% # Den Inhalt der Arbeit einbinden #
% ###################################

\section{Text}

Hier könnt ihr den eigentlichen Text einfügen. \\
Zitate: \cite{anleitung2012} und \cite{anleitung2013}.

% ###################################
% # Literaturverzeichnis mit BibTeX #
% ###################################

% Dies ist die einfachste Variante um ein Literaturverzeichnis zu erzeugen.
% Die einzelnen Einträge müssen dabei per Hand angelegt und formatiert werden.
% Es kann auch bibtex verwendet werden, so wie es in der komplexen Vorlage der Fall ist.

\newpage
% entweder ...
\begin{thebibliography}{2} % https://de.wikibooks.org/wiki/LaTeX-Kompendium:_Schnellkurs:_Erstellen_eines_Literaturverzeichnisses
    \bibitem[1]{anleitung2012} Markus Donath und Anke Schmidt.
        \emph{Anleitung zu den Experimentellen Übungen zur Mechanik und Elektrizitätslehre}, Oktober 2012.
    \bibitem[2]{anleitung2013} Markus Donath und Anke Schmidt.
        \emph{Anleitung zu den Experimentellen Übungen zur Optik, Wärmelehre und Atomphysik}, Oktober 2013.
\end{thebibliography}
% oder ...
%\bibliography{literatur}
%\bibliographystyle{unsrtdin}

% ####################
% # Anhang erstellen #
% ####################

\newpage
\appendix
\section{Anhang}

Hier ist Platz für euren Anhang. Wird kein Anhang benötigt, dann löscht einfach die diese und die 4 vorherigen Zeilen.\\

Den Anhang habe ich absichtlich hinter dem Literaturverzeichnis platziert, so dass man einfach seine handschriftlichen Notizen (z.\,B. die Fehlerrechnung) mit ans Protokoll heften kann.

% #######################
% # Ende des Dokumentes #
% #######################

\end{document}
