\section{Einleitung}
In diesem Versuch wird das Verhalten von Licht, dass auf unterschiedliche Arten von optischen Komponenten manipuliert wird, untersucht.
Obwohl Licht eine Welle ist wird es in diesem Versuch als Strahl angesehen, also Linien im Raum, daher geometrische Optik.
Die verwendeten optischen Komponenten sind: ein Prisma, ein Gitter, eine Sammellinse und eine Streulinse.
Das Prisma besteht aus Flintglas und ist gleichseitig (\alpha=60°).
Für den Ablenkwinkel gilt,
\begin{equation}
\delta(\lambda)=\vartheta+\arcsin [(\sin \alpha)(n(\lambda)^{2}-\sin^{2}\vartheta)^{\frac{1}{2}}-\sin\varthetac \cos\alpha]-\alpha \label{eq:delta}
\end{equation}
Wird der Ablenkwinkel minimal lässt sich daraus der Brechungsindex des Materials bestimmen durch,
\begin{equation}
n = \frac{\sin \frac{\alpha + \delta_m}{2}}{\sin \delta_m} \label{eq:brech_pris}
\end{equation}
Hierbei ist \vartheta der Einfallswinkel, \alpha der Apexwinkel und \delta der Ablenkwinkel, dieser verändert sich außerdem mit der Wellenlänge des Lichts.
Es fällt Licht einer bestimmten Wellenlänge \lambda auf ein Gitter mit Gitterkonstanten \textit{g}=Lückenabstand, es entsteht ein Interferenzmuster.
Es wird der Winkel für das erste Hauptmaxima gemessen, die allgemeine Gleichung lautet:
\begin{equation}
\sin\vartheta_{max}=m\frac{\lambda}{g}
\end{equation}
Dieser Versuch wird einmal mit Luft in der Halbkreisküvette durchgeführt und einmal mit Wasser. Aus der unterschiedlichen Ablenkwinkeln \vartheta_{max} lässt sich der Brechungsindex von Wasser bestimmen, es gilt
\begin{equation}
\frac{\lambda}_{Wasser}{•\lambda_{Luft}}=n_{Wasser} \label{Brechungsindex_wasser}
\end{equation}
Für die nächste optische Komponente, die Linse, gilt nach einigen Approximationen und Rechenschritten die Linsengleichung,
\begin{equation}
\frac{1}{f}=\frac{1}{b}+\frac{1}{g} \label{Linsenglg}
\end{equation}
hierbei ist \textit{f}, die Brennweite, \textit{g} die Gegenstandsweite und \textit{b} die Bildweite.
Für Linsensysteme die aus zwei Einzellinsen mit Brennweiten \textit{f}_{1} und \textit{f}_{2}, gilt für die Gesamtbrennweite,
\begin{equation}
\frac{1}{f}=\frac{1}{f_{1}}+\frac{1}{f_{2}}-\frac{d}{f_{1}f_{2}}
\end{equation}
\textit{d} ist hier der Abstand der Linsen voneinander.
Da die Linsengleichung nur unter sehr eingeschränkten Bedingungen eine ausreichende Genauigkeit hat, wird es in vielen Fällen Abweichungen geben. Diese werden als Linsenfehler bezeichnet.
\newpage
\section{Auswertung}

\newpage
\section{Diskussion} 