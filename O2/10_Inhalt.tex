\section{Einleitung}
Diese Versuchsreihe befasst sich mit den verschiedenen Eigenschaften von Mikrowellenstrahlung. Mikrowellenstrahlung bezeichnet elektromagnetische Wellen im Wellenlängenbereich von etwa \SI{1}{\milli\meter} bis \SI{30}{\centi\meter}. Der Frequenzbereich von Mikrowellen beträgt etwa \SIrange{1}{300}{\giga\hertz}. Mikrowellenstrahlung findet im Alltag Einsatz in der Funkübertragung (zum Beispiel : WLAN, Mobilfunk) und im Mikrowellenherd.
\subsection{Strahldivergenz}
Der Strahlenemitter der in den Versuchen verwendet wird ist konzipiert, um einen möglichst kollimierten Strahl zu emittieren. Jedoch ist perfekte Fokussierung in der Praxis nicht möglich. Stattdessen werden die Mikrowellen unter einem Öffnungswinkel $ \Theta $ ausgesandt. Diesen Winkel bezeichnet man als Strahlendivergenz.
\newpage
\section{Auswertung}

\subsection{Strahlendivergenz}

\subsection{title}
\newpage
\section{Diskussion} 
