\section{Einleitung}
Diese Versuchsreihe befasst sich mit den verschiedenen Eigenschaften von Mikrowellenstrahlung. Mikrowellenstrahlung bezeichnet elektromagnetische Wellen im Wellenlängenbereich von etwa \SI{1}{\milli\meter} bis \SI{30}{\centi\meter}. Der Frequenzbereich von Mikrowellen beträgt etwa \SIrange{1}{300}{\giga\hertz}. Mikrowellenstrahlung findet im Alltag Einsatz in der Funkübertragung (zum Beispiel : WLAN, Mobilfunk) und im Mikrowellenherd.
\subsection{Strahldivergenz}
Der Strahlenemitter der in den Versuchen verwendet wird ist konzipiert, um einen möglichst kollimierten Strahl zu emittieren. Jedoch ist perfekte Fokussierung in der Praxis nicht möglich. Stattdessen werden die Mikrowellen unter einem Öffnungswinkel $ \Theta $ ausgesandt. Diesen Winkel bezeichnet man als Strahlendivergenz.
\newpage
\section{Auswertung}

\subsection{Strahlendivergenz}
\subsection{Wellenlänge}
Um die Wellenlänge zu bestimmen baut man den Versuchsaufbau um, so dass Welle an einer Metallplatte reflektiert wird. Diese steht dabei senkrecht zu Ausbreitungsrichtung der Welle. Die Welle bildet zusammen mit ihrer Reflektion eine stehende Welle. Die Abstand der Intensitätsminima entspricht gemäß \eqref{eq:lambda_stehend} der halben Wellenlänge. \\
Die Intensitätsminima lassen sich mit einer Stabantenne messen. Man verwendet hier nicht den zuvor verwendeten Detektor, da dieser die Welle zu einem deutlich größeren Teil absorbieren Würde und somit das Modell der stehenden Welle nicht mehr gelten würde. \\
Bei der Versuchsdurchführung hat sich gezeigt, dass abweichend vom theoretischen Modell der stehenden Welle die Intensität in den Wellenknoten nicht $ 0 $ wird. Dennoch sind ab einem gewissen Abstand vom Emitter Maxima und Minima deutlich messbar. Wir fanden Minima in den Abständen \SIlist{14+-.1; 15,5+-.1; 17,2+-.1; 19+-.1; 20,2+-.1}{\centi\meter} vom Reflektor.
\subsection{}
\subsection{title}
\newpage
\section{Diskussion} 
