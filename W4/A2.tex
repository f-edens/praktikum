%% LyX 2.1.2 created this file.  For more info, see http://www.lyx.org/.
%% Do not edit unless you really know what you are doing.
\documentclass[twoside,ngerman]{scrartcl}
\usepackage{mathpazo}
\usepackage[T1]{fontenc}
\usepackage[utf8]{luainputenc}
\usepackage[a4paper]{geometry}
\geometry{verbose,tmargin=2cm,bmargin=25mm,lmargin=20mm,rmargin=10mm}
\usepackage{fancyhdr}
\pagestyle{fancy}
\usepackage{babel}
\usepackage{amsmath}
\usepackage[unicode=true,pdfusetitle,
 bookmarks=true,bookmarksnumbered=true,bookmarksopen=false,
 breaklinks=false,pdfborder={0 0 1},backref=false,colorlinks=false]
 {hyperref}

\makeatletter
%%%%%%%%%%%%%%%%%%%%%%%%%%%%%% Textclass specific LaTeX commands.
\numberwithin{equation}{section}

%%%%%%%%%%%%%%%%%%%%%%%%%%%%%% User specified LaTeX commands.
\usepackage{pgfplots}
\pgfplotsset{width=7cm}

\makeatother

\begin{document}

\title{Versuchsprotokoll}


\subtitle{Versuch \{Versuchsnummer\}:\\
\{Versuchstitel\}}


\date{\{Datum\}}


\author{Gruppe 6MO:\\
Frederik Edens\\
Dennis Eckermann}

\maketitle
\vfill{}


\tableofcontents{}

\vfill{}


\newpage{}


\section{Einleitung}


\section{Versuchsteil}


\subsection{Messung der molaren Masse anhand der Dampfdichtemethode}


\subsubsection{Auswertung}


\subsection{Messung der molaren Masse anhand er Gefrierpunktserniedrigung}

In diesem Teil wird im Lösungsmittel Cyclohexan eine geringe Menge
Eicosan gelöst und der Unterschied der Gefrierpunkte gemessen. Dabei
liegen die Mengen bei $\left(16,27\pm0,02\right)g$ Cyclohexan und
$\left(0,39\pm0,01\right)g$ Eicosan.

Zunächst wurde der Gefrierpunkt des Cyclohexans gemessen. Dafür wurde
die Subtanz in einem Reagenzglas in ein Eisbad gelegt und unter ständigem
rühren abgekühlt. Wird ein Plateau erreicht, wurde der Gefrierpunkt
ermittelt. Dieser liegt bei $T_{1}=\left(6,6\pm0,1\right)\text{°C}$. 

Die Ermittlung des Gefrierpunkts für das Stoffgemisch aus Cyclohexan
und Eicosan ist analog, nur ist das rühren noch wichtiger als vorher.
Hier liegt der Gefrierpunkt bei $T_{2}=\left(3,3\pm0,1\right)\text{°C}$.

Damit beträgt die Temperaturdifferenz $\triangle T=T_{1}-T_{2}=\left(3,3\pm0,2\right)K$.

Nach (1.8) gilt,
\[
M_{S}=20,2\cdot10^{3}\frac{gK}{mol}\frac{1}{\left(16,27\pm0,02\right)g}\frac{\left(0,39\pm0,01\right)g}{\left(3,3\pm0,2\right)K}
\]


das ergibt,
\[
M_{S}=\left(146,73\pm4,80\right)\frac{g}{mol}
\]


dieser Wert stimmt mit dem Literaturwert von $282,5\frac{g}{mol}$
leider nicht überein. 


\section{Diskussion}

Das Ergebnis für die Gefrierpunktserniedrigung weicht weit vom Literaturwert
ab. Generell wurden alle Messungen korrekt durchgeführt, aber um diesen
Fehler zu erklären muss es an einer Stelle einen groben Fehler gegeben
haben. Die Temperaturdifferenz ist korrekt und das Eicosan war bereits
in 400mg Rationen aufgeteilt, somit ist es wahrscheinlich, dass der
Fehler bei der Menge des Cyclohexans liegt. Wobei zu erwähnen ist,
dass dieser Wert auch im Rahmen liegt. +
\end{document}
