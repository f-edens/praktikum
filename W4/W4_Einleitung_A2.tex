%% LyX 2.1.2 created this file.  For more info, see http://www.lyx.org/.
%% Do not edit unless you really know what you are doing.
\documentclass[twoside,ngerman]{scrartcl}
\usepackage{mathpazo}
\usepackage[T1]{fontenc}
\usepackage[utf8]{luainputenc}
\usepackage[a4paper]{geometry}
\geometry{verbose,tmargin=2cm,bmargin=25mm,lmargin=20mm,rmargin=10mm}
\usepackage{fancyhdr}
\pagestyle{fancy}
\usepackage{babel}
\usepackage{amsmath}
\usepackage[unicode=true,pdfusetitle,
 bookmarks=true,bookmarksnumbered=true,bookmarksopen=false,
 breaklinks=false,pdfborder={0 0 1},backref=false,colorlinks=false]
 {hyperref}

\makeatletter
%%%%%%%%%%%%%%%%%%%%%%%%%%%%%% Textclass specific LaTeX commands.
\numberwithin{equation}{section}

%%%%%%%%%%%%%%%%%%%%%%%%%%%%%% User specified LaTeX commands.
\usepackage{pgfplots}
\pgfplotsset{width=7cm}

\makeatother

\begin{document}


\newpage{}


\section{Einleitung}

In diesem Versuch werden molare Massen mittels zwei verschiedener
Methoden bestimmt.

Einmal mit der Dampfdichtemethode, das andere mal mit der Gefrierpunktserniedrigung.

Zunächst ist als molare Masse das Verhältnis der Masse und der Stoffmenge
definiert,
\begin{equation}
M=\frac{m}{\nu}\frac{g}{mol}\label{eq:Molmasse}
\end{equation}


wobei $\nu$ die Stoffmenge ist und m die Masse.

Ein Mol ist definiert als die Anzahl der Teilchen die in 12g des Kohlenstoffisotops
$^{12}C$ enthalten sind.

Das Molvolumen ist definiert durch das Volumen durch die Stoffmenge
\begin{equation}
V_{m}=\frac{V}{\nu}=\frac{M}{\varrho}\frac{m^{3}}{mol}\label{eq:Molvolumen1}
\end{equation}


V ist das Volumen und $\varrho=\frac{m}{V}$ die Dichte des Stoffes.

Aus der idealen Gasgleichung folgt für eine Stoffmenge von einem Mol.

\begin{equation}
V_{m}=\frac{RT}{p}\label{eq:Molvolumen2}
\end{equation}


R ist die allgemeine Gaskonstante und hat den Wert 8,314$\frac{J}{mol\cdot K}$und
T ist die Temperatur.

Unter Normalbedingungen beträgt das molare Volumen eines idealen Gases
bei einem Mol Stoffmenge etwa 22,41$\frac{l}{mol}$. 

Für eine Stoffmenge von einem Mol ergibt sich für \ref{eq:Molvolumen1}
\begin{equation}
\frac{M}{V_{m0}}=\frac{m}{V_{0}}\label{eq:}
\end{equation}


durch die ideale Gasgleichung bei Normalbedingungen und einigen weiteren
Schritten ergibt sich,
\begin{equation}
M=m\frac{V_{mo}}{V}\frac{p_{0}}{p}\frac{T}{T_{0}}\label{eq:Formel}
\end{equation}


diese wird in der weiteren Auswertung gebraucht, m ist die Masse der
Probesubstanz $p_{0}$und $T_{0}$ sind Druck und Temperatur unter
Normalbedingungen.

Da die Probesubstanz mit einer Spritze aufgezogen wird, ist diese
nicht direkt Messbar. Es ist nur die Differenz der Massen der jeweils
gefüllten und leeren Spritzen bekannt.

Es kommt noch ein Zusatzterm hinzu, der dem Auftrieb zu verschulden
ist, somit folgt
\begin{equation}
m=\left(m_{2}-m_{1}\right)+\varrho_{L}V_{Fl}\label{eq:masse}
\end{equation}


$\varrho_{L}$ ist die Dichte der Luft und $V_{Fl}$ das Volumen der
Probesubstanz.

Damit sind die grundlegenden Formeln für die Dampfdichtemethode besprochen,
zur funktionsweise der Apperatur mehr im Versuchsteil.

Bei der Gefrierpunktserdniedrigung wird die molare Masse, durch die
Änderung des Gefrierpunkts eines Stoffes, in dem ein anderer Stoff
in diesem gelöst wird, bestimmt.

Diese wird durch folgende Formel beschrieben,

\begin{equation}
\triangle T=K\frac{1}{m_{L}}\frac{m_{S}}{M_{S}}\label{eq:Temp_diff}
\end{equation}


K heißt kryoskopische Konstante und ist für Lösungsmittel die charakteristische
Größe, in diesem Fall ist das Lösungsmittel Cyclohexan. Hierfür hat
K den Wert $20,2\cdot10^{3}\frac{gK}{mol}$. $m_{L}$ist die Masse
des Lösungsmittels $m_{S}$ die Masse der gelösten Substanz und $M_{S}$ist
die molare Masse der gelösten Substanz.

Diese lässt sich durch einfaches umstellen der Formel berechnen,
\begin{equation}
M_{S}=K\frac{1}{m_{L}}\frac{m_{S}}{\triangle T}\label{eq:molareMasse}
\end{equation}


damit lassen sich die gewünschten Berechnungen durchführen.


\section{Versuchsteil}


\subsection{Messung der molaren Masse anhand der Dampfdichtemethode}


\subsubsection{Auswertung}


\subsection{Messung der molaren Masse anhand er Gefrierpunktserniedrigung}

In diesem Teil wird im Lösungsmittel Cyclohexan eine geringe Menge
Eicosan gelöst und der Unterschied der Gefrierpunkte gemessen. Dabei
liegen die Mengen bei $\left(16,27\pm0,02\right)g$ Cyclohexan und
$\left(0,39\pm0,01\right)g$ Eicosan.

Zunächst wurde der Gefrierpunkt des Cyclohexans gemessen. Dafür wurde
die Subtanz in einem Reagenzglas in ein Eisbad gelegt und unter ständigem
rühren abgekühlt. Wird ein Plateau erreicht, wurde der Gefrierpunkt
ermittelt. Dieser liegt bei $T_{1}=\left(6,6\pm0,1\right)\text{°C}$. 

Die Ermittlung des Gefrierpunkts für das Stoffgemisch aus Cyclohexan
und Eicosan ist analog, nur ist das rühren noch wichtiger als vorher.
Hier liegt der Gefrierpunkt bei $T_{2}=\left(3,3\pm0,1\right)\text{°C}$.

Damit beträgt die Temperaturdifferenz $\triangle T=T_{1}-T_{2}=\left(3,3\pm0,2\right)K$.

Nach (1.8) gilt,
\[
M_{S}=20,2\cdot10^{3}\frac{gK}{mol}\frac{1}{\left(16,27\pm0,02\right)g}\frac{\left(0,39\pm0,01\right)g}{\left(3,3\pm0,2\right)K}
\]


das ergibt,
\[
M_{S}=\left(146,73\pm4,80\right)\frac{g}{mol}
\]


dieser Wert stimmt mit dem Literaturwert von $282,5\frac{g}{mol}$
leider nicht überein. 


\section{Diskussion}

Das Ergebnis für die Gefrierpunktserniedrigung weicht weit vom Literaturwert
ab. Generell wurden alle Messungen korrekt durchgeführt, aber um diesen
Fehler zu erklären muss es an einer Stelle einen groben Fehler gegeben
haben. Die Temperaturdifferenz ist korrekt und das Eicosan war bereits
in 400mg Rationen aufgeteilt, somit ist es wahrscheinlich, dass der
Fehler bei der Menge des Cyclohexans liegt. Wobei zu erwähnen ist,
dass dieser Wert auch im Rahmen liegt. 
\end{document}
