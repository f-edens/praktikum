% Der Befehl \newcommand kann auch benutzt werden um Variablen zu definieren:

% Nummer laut Praktikumsheft:
    \newcommand{\varNum}{O6}
% Name laut Praktikumsheft:
    \newcommand{\varName}{Beugung}
% Datum der Durchführung:
    \newcommand{\varDate}{29. Juni 2015}
% Autoren des Protokolls:
    \newcommand{\varAutor}{Frederik Edens, Dennis Eckermann}
% Nummer der eigenen Gruppe (z.B. "1mo"):
    \newcommand{\varGruppe}{Gruppe 6mo}
% E-Mail-Adressen der Autoren:
    \newcommand{\varEmail}{\url{fr.ed@wwu.de}\\\url{dennis.eckermann@gmx.de}}
% E-Mail-Adresse anzeigen (true/false):
    \newcommand{\varZeigeEmail}{true}
% Literaturverzeichnis anzeigen (true/false):
    \newcommand{\varZeigeLiteraturverzeichnis}{true}
% Stil der Einträge im Literaturverzeichnis
    \newcommand{\varLiteraturLayout}{unsrtdin}
