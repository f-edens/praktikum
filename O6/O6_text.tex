\section{Auswertung}
\subsection{Doppelspalt und Einzelspalt}
Werden die Intensitätsverteilung von einem Einzelspalt und einem Doppelspalt mit gleicher Spaltbreite $ b= 0,015\SI{0,015}{\milli}{\meter} $ so ist direkt sichtbar, dass die ersten Hauptmaxima des Doppelspalts deutlich höher in ihrer Intensität sind. Das nullte Hauptmaxima ist deutlich schärfer als das des Einzelspalts. Die Hauptmaxima höherer Ordnung haben ebenfalls eine höhere Intensität.
\subsection{Doppelspalte}
Nun werden Doppelspalte mit unterschiedlichen Spaltabständen $ ( 0,25mm,0,5mm und 1mm) $ untersucht.
Während bei $ 0,25mm $ und $ 0,5mm $ starke Interferenzphänomene beobachtbar sind, ist bei einem Spaltabstand von $ 1mm $ kaum Interferenz sichtbar.
\section{Diskussion}
\subsection{Doppelspalte}
Die stärksten Interferenzphänomene sind bei einem Spaltabstand von $ 0,5mm $ beobachtbar. Bei einem Abstand von $ 0,25mm $ ist ebenfalls Interferenz zu sehen, bei einem Abstand von $ 1mm $ aber war die Intensitätsverteilung eher die eines Einzelspalts. Wahrscheinlich ist dieser Abstand zu groß um Interferenz zu erzeugen.
\subsection{Mehrfachspalte}
