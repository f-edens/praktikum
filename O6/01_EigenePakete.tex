% Die Folgenden Pakete sind schon eingebunden (siehe 00_Protokoll.tex):
% \usepackage[utf8x]{inputenc}                             % Legt die Zeichenkodierung fest, z.B UTF8
% \usepackage[T1]{fontenc}                                 % Verwendung der Zeichentabelle T1, für deutschsprachige Dokumente sinnvoll
% \usepackage[ngerman,english]{babel}                      % Silbentrennung nach neuer deutscher und englischer Rechtschreibung
% \usepackage{amsmath}                                     % Mathepaket
% \usepackage{ifthenx}                                     % Wird benötigt um \ifthenelse zu benutzen
% \usepackage[pdftex]{graphicx}                            % Zum flexiblen Einbinden von Grafiken, pdftex ist optional
% \usepackage{units}                                       % Ermöglicht die Nutzung von \unit[Zahl]{Einheit}
% \usepackage{setspace}                                    % Einfaches wechseln zwischen unterschiedlichen Zeilenabständen
% \usepackage[pdfpagelabels]{hyperref}                     % Verlinkt Textstellen im PDF Dokument
% \usepackage[font=small,labelfont=bf,labelsep=endash,format=plain]{caption}
%                                                          % Darstellung für Caption s.u.
% \usepackage{subfig}                                      % Bilder nebeneinander
% \usepackage{wrapfig}                                     % Fließtext um Figure-Umgebung
% \usepackage{cite}                                        % Zusatzfunktionen zum zitieren
% \usepackage{scrpage2}                                    % Wird für Kopf- und Fußzeile benötigt
% \usepackage{array,dcolumn}                               % Beide Pakete werden für die Ausrichtung der Tabellenspalten benötigt

% Pakete erweitern LaTeX um zusätzliche Funktionen. Dies ist eine Satz nützlicher Pakete.
% Weitere sollten in der Datei"`01_EigenePakete.tex"' hinzugefügt werden.
\usepackage[utf8]{inputenc}                                                  % Legt die Zeichenkodierung fest, z.B UTF8
\usepackage[T1]{fontenc}                                                      % Verwendung der Zeichentabelle T1, für deutschsprachige Dokumente sinnvoll
\usepackage[ngerman,english]{babel}                                           % Silbentrennung nach neuer deutscher und englischer Rechtschreibung
\usepackage{amsmath}                                                          % Mathepaket
\usepackage{xifthen}                                                          % Wird benötigt um \ifthenelse zu benutzen
\usepackage[pdftex]{graphicx}                                                 % Zum flexiblen Einbinden von Grafiken, pdftex ist optional
\usepackage{units}                                                            % Ermöglicht die Nutzung von \unit[Zahl]{Einheit}
\usepackage{setspace}                                                         % Einfaches wechseln zwischen unterschiedlichen Zeilenabständen
\usepackage[pdfpagelabels]{hyperref}                                          % Verlinkt Textstellen im PDF Dokument
\usepackage[font=small,labelfont=bf,labelsep=endash,format=plain]{caption}    % Darstellung für Caption s.u.
\usepackage{subfig}                                                           % Bilder nebeneinander
\usepackage{wrapfig}                                                          % Fließtext um Figure-Umgebung
%\usepackage{cite}                                                             % Zusatzfunktionen zum zitieren
\usepackage{scrpage2}                                                         % Wird für Kopf- und Fußzeile benötigt
\usepackage{array,dcolumn}                                                    % Beide Pakete werden für die Ausrichtung der Tabellenspalten benötigt



% weitere Pakete einbinden

% % % % % % % % % % % % % % % % % % % % % %
% % Eigene

\usepackage{csquotes}
\usepackage[backend=biber, style=numeric]{biblatex}
\addbibresource{literatur.bib} %Bibliographie

\usepackage[locale=DE]{siunitx}
\sisetup{separate-uncertainty, list-final-separator = {\text{ und }}}
\usepackage{pgfplots}
%\pgfplotsset{/pgf/number format/use comma}
\SendSettingsToPgf
\pgfplotsset{compat=1.9}
%\usepgfplotslibrary{external} 
%\tikzexternalize

\usepackage{gnuplottex}
\usepackage{longtable}
\usepackage{isotope}
\usepackage{float}