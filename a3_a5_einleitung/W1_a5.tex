%% LyX 2.1.2 created this file.  For more info, see http://www.lyx.org/.
%% Do not edit unless you really know what you are doing.
\documentclass[twoside,ngerman]{scrartcl}
\usepackage{mathpazo}
\usepackage[T1]{fontenc}
\usepackage[latin9]{inputenc}
\usepackage[a4paper]{geometry}
\geometry{verbose,tmargin=2cm,bmargin=25mm,lmargin=20mm,rmargin=10mm}
\usepackage{fancyhdr}
\pagestyle{fancy}
\usepackage{babel}
\usepackage{amsmath}
\usepackage[unicode=true,pdfusetitle,
 bookmarks=true,bookmarksnumbered=true,bookmarksopen=false,
 breaklinks=false,pdfborder={0 0 1},backref=false,colorlinks=false]
 {hyperref}
\usepackage{breakurl}

\makeatletter
%%%%%%%%%%%%%%%%%%%%%%%%%%%%%% Textclass specific LaTeX commands.
\numberwithin{equation}{section}

%%%%%%%%%%%%%%%%%%%%%%%%%%%%%% User specified LaTeX commands.
\usepackage{pgfplots}
\pgfplotsset{width=7cm}

\makeatother

\begin{document}

\title{Versuchsprotokoll}

\maketitle
\tableofcontents{}

\newpage{}

Um den Wirkungsgrad durch Abbremsen zu bestimmen, wird eine Reibungsbremse
benutzt, dem Prony'schen Zaum. Dieser kann an das Schwungrad befestigt
werden und mithilfe von zwei Schrauben je nach n�tiger Reibung einstellbar.
Bei �nderung der Reibungskraft, ver�ndert sich auch die Frequenz mit
der das System schwingt.

Das Schwungrad �bt ein Drehmoment auf den Zaum aus, dieser wird mithilfe
einer Federwaage gemessen. Das Federpendel l�sst allerdings nur Messungen
bis zu 1N zu, daher sind zus�tzlich noch Gewichte vorhanden die den
Zaum nochmal mit 0,5N ,,entlasten``, damit Messungen >1N m�glich
sind.

Bei dieser Messreihe war es nicht m�glich ein Drehmoment von �ber
0,75N an der Federwaage plus ein Zusatzgewicht zu erreichen (=1,25N),
da der der Stirlingmotor dann so stark gebremst wurde, dass er nicht
mehr schwang. 

Der Wirkungsgrad wird aus dem Quotienten von mechanischer Leistung
geteilt durch die zugef�hrte elektrische Leistung ermittelt.
\[
\eta=\frac{P_{Zaum}}{P_{elektrisch}}=\frac{2\pi fFr}{UI}
\]


f�r den Fehler gilt nach Gau�'scher Fehlerfortpflanzung
\[
\triangle\eta=\sqrt{\left(\frac{\partial\eta}{\partial P_{Zaum}}\triangle P_{Zaum}\right)^{2}+\left(\frac{\partial\eta}{\partial P_{elektrisch}}\triangle P_{elektrisch}\right)^{2}}
\]


Die elektrische Leistung betr�gt ( wie in Teil 4 ermittelt) 2250,6W.
Der Wirkungsgrad liegt hier in einem Bereich von $10^{-4}$vor, ist
also ziemlich gering.

Der maximale Wirkungsgrad liegt hier bei einer Frequenz von 3,4Hz
vor.

Auch liegt hier eine Schwankung vor die nicht in den allgemeinen Trend
passt, n�mlich zwischen 4Hz und 3,5Hz sinkt der Wirkungsgrad wieder
etwas, wahrscheinlich wurde ein Messfehler begangen, indem die Federwaage
oder der Zaum nicht richtig eingestellt waren.

\begin{figure}
\protect\caption{Wirkungsgrad gegen Frequenz}
\end{figure}

\end{document}
