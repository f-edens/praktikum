%% LyX 2.1.2 created this file.  For more info, see http://www.lyx.org/.
%% Do not edit unless you really know what you are doing.
\documentclass[twoside,ngerman]{scrartcl}
\usepackage{mathpazo}
\usepackage[T1]{fontenc}
\usepackage[latin9]{inputenc}
\usepackage[a4paper]{geometry}
\geometry{verbose,tmargin=2cm,bmargin=25mm,lmargin=20mm,rmargin=10mm}
\usepackage{fancyhdr}
\pagestyle{fancy}
\usepackage{babel}
\usepackage{textcomp}
\usepackage{amsmath}
\usepackage{graphicx}
\usepackage[unicode=true,pdfusetitle,
 bookmarks=true,bookmarksnumbered=true,bookmarksopen=false,
 breaklinks=false,pdfborder={0 0 1},backref=false,colorlinks=false]
 {hyperref}
\usepackage{breakurl}

\makeatletter
%%%%%%%%%%%%%%%%%%%%%%%%%%%%%% Textclass specific LaTeX commands.
\numberwithin{equation}{section}

%%%%%%%%%%%%%%%%%%%%%%%%%%%%%% User specified LaTeX commands.
\usepackage{pgfplots}
\pgfplotsset{width=7cm}

\makeatother

\begin{document}

\title{Versuchsprotokoll}


\subtitle{Versuch \{Versuchsnummer\}:\\
\{Versuchstitel\}}


\date{\{Datum\}}


\author{Gruppe 6MO:\\
Frederik Edens\\
Dennis Eckermann}

\maketitle
\vfill{}


\tableofcontents{}

\vfill{}


\newpage{}

Zur Bestimmung der Heizleistung wird nun der Motor als W�rmepumpe
betrieben, dazu wird die Umlaufrichtung umgekehrt. In Abbildung 1,
ist der Temperaturverlauf in Abh�ngigkeit von der Zeit dargestellt.
Am Anfang der Messung, ist f�r die ersten Sekunden kein Anstieg zu
sehen, das liegt daran, dass diese Messung unmittelbar durchgef�hrt
wurde, nachdem der Motor als K�ltemaschine benutzt wurde, daher wurden
einige Umdrehungen ben�tigt um den Aufheizprozess einzuleiten. Danach
ist ein relativer linearer Anstieg der Temperatur zu sehen, ab einer
Temperatur von ca. 0\textdegree C ist ein Plateu zu sehen, die Erkl�rung
hierf�r ist, dass das Eis nun am schmelzen ist, dies bedeutet, dass
die zugef�hrte W�rmeenergie nicht in einer Temperaturerh�hung resultiert,
sondern f�r den Schmelzvorgang ben�tigt wird. Ist der Schmelzprozess
abgeschlossen steigt die Temperatur Erwartungsgem�� wieder linear
an.

Um nun die spezifische W�rme von Eis zu bestimmen, wird zuerst angenommen,
dass die zugef�hrte W�rmeenergie pro Zeit konstant ist, da au�erdem
die spezifische W�rme von Wasser bekannt ist $\left(4,185Jg^{-1}K^{-1}\right)$,
die Masse des Wassers ist in guter N�herung mit $(1\pm0,1)$g approximiert.
Nachdem der Schmelzprozess beendet ist, was bei ca. 8,2\textdegree C
bei einer Zeit von 230s der Fall ist, wird der weitere Verlauf als
linear angenommen. Am Ende der Messung hat das Wasser eine Temperatur
von 49,1\textdegree C bei der Zeit von 375s, daraus ergibt sich eine
Temperaturerh�hung pro Zeit von$0,282\frac{K}{s}$. Die gesamte zugef�hrte
W�rmeenergie l�sst sich auch berechnen (die spezifische W�rme von
Wasser und die Temperatur werden als genau angenommen)
\[
(1\pm0,1)g\cdot4,185\frac{J}{gK}\cdot40,9K=(171,16\pm17,11)J
\]


daraus ergibt sich eine Energiezufuhr von$(1,20\pm0,12)\frac{J}{s}$.

Das Eis wird in einer Zeit von 53s um 24K erw�rmt, also gilt
\[
(1\pm0,1)g\cdot c_{Eis}\cdot24K=(63,6\pm6,4)J
\]


also f�r die (abgesch�tzte) spezifische W�rme
\[
c_{Eis}=(2,650\pm0,37)\frac{J}{gK}
\]


f�r den Fehler gilt
\[
\triangle c_{Eis}=\sqrt{\left(\frac{\partial c_{Eis}}{\partial m}\triangle m\right)^{2}+\left(\frac{\partial c_{Eis}}{\partial\triangle T}\triangle\triangle T\right)^{2}+\left(\frac{\partial c_{Eis}}{\partial\triangle Q}\triangle\triangle Q\right)^{2}}
\]


dieser Wert stimmt nicht mit dem Literaturwert von $2,060\frac{J}{gK}$
(Quelle: https://de.wikipedia.org/wiki/Spezifische\_W/C3/A4rmekapazit/C3/A4t
) �berein, zum einen kann es daran liegen, dass es sich hier um eine
Absch�tzung handelt und einige Annahmen getroffen wurden die das Ergebnis
verf�lschen. Au�erdem k�nnen gewisse Daten nicht mit 100\%iger Genauigkeit
gesch�tzt werden was zu Fehlern f�hren kann!

Die Temperatur des Wasser-Reservoirs ist konstant geblieben, bei 23,1\textdegree C
und die Temperatur des K�hlwassers ist von 24,3\textdegree C auf 22,9\textdegree C
abgesunken!

\begin{figure}
\protect\caption{Wassertemperatur gegen Zeit}
\includegraphics{\string"C:/Users/hp/Desktop/Praktikum/Praktikum SS15/praktikum/W1/diagramme/a2_temp\string".eps}
\end{figure}

\end{document}
