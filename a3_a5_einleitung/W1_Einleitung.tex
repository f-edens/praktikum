%% LyX 2.1.2 created this file.  For more info, see http://www.lyx.org/.
%% Do not edit unless you really know what you are doing.
\documentclass[twoside,ngerman]{scrartcl}
\usepackage{mathpazo}
\usepackage[T1]{fontenc}
\usepackage[latin9]{inputenc}
\usepackage[a4paper]{geometry}
\geometry{verbose,tmargin=2cm,bmargin=25mm,lmargin=20mm,rmargin=10mm}
\usepackage{fancyhdr}
\pagestyle{fancy}
\usepackage{babel}
\usepackage{amsmath}
\usepackage{esint}
\usepackage[unicode=true,pdfusetitle,
 bookmarks=true,bookmarksnumbered=true,bookmarksopen=false,
 breaklinks=false,pdfborder={0 0 1},backref=false,colorlinks=false]
 {hyperref}
\usepackage{breakurl}

\makeatletter
%%%%%%%%%%%%%%%%%%%%%%%%%%%%%% Textclass specific LaTeX commands.
\numberwithin{equation}{section}

%%%%%%%%%%%%%%%%%%%%%%%%%%%%%% User specified LaTeX commands.
\usepackage{pgfplots}
\pgfplotsset{width=7cm}

\makeatother

\begin{document}

\title{Versuchsprotokoll}


\subtitle{Versuch \{Versuchsnummer\}:\\
\{Versuchstitel\}}


\date{\{Datum\}}


\author{Gruppe 6MO:\\
Frederik Edens\\
Dennis Eckermann}

\maketitle
\vfill{}


\tableofcontents{}

\vfill{}


\newpage{}


\section{Einleitung}

Die Versuchsreihe befasst sich ausschlie�lich mit dem Stirling-Motor.
Als theoretische Grundlagen dazu sind insbesondere Kenntnisse von
Temperatur und W�rmemenge, thermodynamischen Zust�nden und Zustandsgleichungen
sowie Kreisprozessen notwendig.


\subsection{Temperatur und W�rmemenge und weitere Thermodynamik}

Im wesentlichen ist der Unterschied zwischen W�rmemenge und Temperatur,
dass die W�rmemenge die Masse und die spezifische W�rme mit einbezieht,
also einen Materialabh�ngigen Wert, das hei�t, dass je nach Beschaffenheit
des Materials zwei Stoffe die gleiche Temperatur und die gleiche Masse
haben, dennoch eine W�rmemenge (Energie) in sich tragen und somit
unterschiedlich viel Arbeit verrichten k�nnen.

\begin{equation}
c\cdot m\cdot\triangle T=\triangle Q\label{eq:W=0000E4rmemenge}
\end{equation}


Nach dem 1. Hauptsatz der Thermodynamik l�sst sich folgende Gleichung
formulieren,
\begin{equation}
Q=\triangle U-W\label{eq:}
\end{equation}


$\triangle U$ ist die �nderung der inneren Energie eines thermodynamischen
Systems, bei einem idealen Gas also die �nderung der kinetischen Energie
der Gasatome, und W ist die verrichtete mechanische Arbeit.

Diese l�sst sich f�r einen isothermen Prozess wie folgt berechnen,

\begin{equation}
W_{isotherm}=-\intop_{V1}^{V2}\frac{nRT}{V}dV\label{eq:mechanische Arbeit}
\end{equation}


dies ist eine N�herung f�r ideale Gase, n ist die Anzahl der Mole
des Gases, R ist die allgemeine Gaskonstante, T ist die Temperatur,
die bei isothermen-Prozessen konstant ist.

Der Wirkungsgrad einer W�rmekraftmaschine ist definiert als geleistete
Arbeit durch die zugef�hrte W�rme,
\begin{equation}
\eta=\frac{|W|}{|Q|}\label{eq:Wirkungsgrad}
\end{equation}


es gibt allerdings keine Maschine die einen Wirkungrad von 100\% erreicht.

Im Gegensatz zum Wirkungsgrad wird die Leistungszahl definiert, diese
gilt f�r K�ltemaschinen bzw. W�rmepumpen, es wird mechanische Arbeit
aufgewendet um eine W�rmemenge zu �bertragen, es gilt:
\begin{equation}
\epsilon=\frac{|Q|}{|W|}\label{eq:Leistungszahl}
\end{equation}



\subsection{Stirling-Kreisprozess}

Der idealisierte Stirling-Kreisprozess besteht aus vier Schritten.
Im ersten Schritt wird bei isothermer Ausdehnung mechanische Arbeit
vom System abgegeben, w�hrend ihm eine gleich gro�e W�rmemenge zugef�hrt
wird. Im zweiten Schritt gibt das System durch isochore Dekompression
Energie in Form von W�rme ab. Im dritten Schritt wird das System isotherm
auf das Ausgangsvolumen zur�ckgebracht. Dabei nimmt es die gleiche
Arbeit auf, wie es an w�rme ab gibt, jedoch weniger als in Schritt
2. Der letzte Schritt ist die isochore Erw�rmung, dabei wird der Verdr�ngerkolben
nach oben bewegt und die Luft nimmt W�rmeenergie auf.

Wie f�r alle reversiblen Kreisprozesse ist auch hier die obere Schranke
des Wirkunsgrades der Carnot-Wirkungsgrad 
\begin{equation}
\eta_{Carnot}=1-\frac{T_{Kalt}}{T_{Warm}}\label{eq:Carnot}
\end{equation}


dieser wird allerdings niemals von realen Kreisprozessen erreicht,
da es sich hier um eine idealisierte Ansicht handelt.
\end{document}
