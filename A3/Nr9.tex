%% LyX 2.1.2 created this file.  For more info, see http://www.lyx.org/.
%% Do not edit unless you really know what you are doing.
\documentclass[english]{article}
\usepackage[T1]{fontenc}
\usepackage[latin9]{inputenc}
\usepackage{babel}
\begin{document}
Im letzten Versuchsteil wurde das $\gamma$-Pr�perat vor das Z�hlrohr
gestellt um, deren Zerf�lle zu bestimmen.

Anhand von Abbildung 5 ist durch die logarithmische Darstellung der
Impulsrate eine lineare Abnahme der Impulse mit der Absorberdicke
sichtbar. W�rde die Impulsrate auf einer linearen Achse dargestellt
werden, so w�re eine exponentielle Abnahme zu sehen.

\begin{figure}


\protect\caption{$\gamma-Strahlen$ gegen Blei}
hier Diagramm einf�gen!
\end{figure}


Um daraus den Absorptionskoeffizienten $\mu$zu bestimmen, kann die
Formel f�r die $\gamma-$Absorption einfach nach $\mu$aufgel�st werden.
Es folgt:

\[
\mu=\frac{ln\left(\frac{N_{0}}{N(x)}\right)}{x}=-0,10837785\frac{1}{mm}
\]


f�r den Massenabsorptionskoeffizienten $\mu_{m}$folgt daraus,
\[
\mu_{m}=\frac{\mu}{\rho_{Blei}}=\frac{-0,10837785\frac{1}{mm}}{0,011342g/mm^{3}}=-9,55\frac{mm^{2}}{g}
\]


diese Werte gelten f�r $\gamma-$Quanten die eine Energie von 0,66MeV
haben!
\end{document}
