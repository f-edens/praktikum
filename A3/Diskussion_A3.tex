%% LyX 2.1.2 created this file.  For more info, see http://www.lyx.org/.
%% Do not edit unless you really know what you are doing.
\documentclass[twoside,ngerman]{scrartcl}
\usepackage{mathpazo}
\usepackage[T1]{fontenc}
\usepackage[utf8]{luainputenc}
\usepackage[a4paper]{geometry}
\geometry{verbose,tmargin=2cm,bmargin=25mm,lmargin=20mm,rmargin=10mm}
\usepackage{fancyhdr}
\pagestyle{fancy}
\usepackage{babel}
\usepackage{amsmath}
\usepackage[unicode=true,pdfusetitle,
 bookmarks=true,bookmarksnumbered=true,bookmarksopen=false,
 breaklinks=false,pdfborder={0 0 1},backref=false,colorlinks=false]
 {hyperref}

\makeatletter
%%%%%%%%%%%%%%%%%%%%%%%%%%%%%% Textclass specific LaTeX commands.
\numberwithin{equation}{section}

%%%%%%%%%%%%%%%%%%%%%%%%%%%%%% User specified LaTeX commands.
\usepackage{pgfplots}
\pgfplotsset{width=7cm}

\makeatother

\begin{document}

\title{Versuchsprotokoll}


\subtitle{Versuch \{Versuchsnummer\}:\\
\{Versuchstitel\}}


\date{\{Datum\}}


\author{Gruppe 6MO:\\
Frederik Edens\\
Dennis Eckermann}

\maketitle
\vfill{}


\tableofcontents{}

\vfill{}


\newpage{}


\section{Einleitung}


\section{Durchführung}


\section{Diskussion}

Im allgemeinen sind die Ergebnisse diese Versuchs mit den Erwartungen
im Einklang. Die Abweichungen von der errechneten Poisson-Verteilung
sind sehr gering und somit vertretbar, bei einer größeren Anzahl von
Messungen wäre eine größere Übereinstimmung zu erwarten.

Bei der $\beta$-Strahlung wurde die ,,mittlere`` Reichweite bei
ca. 520$\mu m$ Aluminum gemessen, da ab dieser Dicke nur noch ca.
50\% der Ausgangsstrahlung gemessen werden konnte.

Der exponentielle Abfall der Impulsrate wie in Abbildung 3 sichtbar
ist, gilt nur näherungsweise bei kleinen Absoberdicken, diese sind
in diesem Fall gegeben und die Messwerte stimmen mit einem exponentiellem
Abfall gut überein.

Wird nun Aluminium mit anderen Materialien verglichen, in diesem Fall
Gummi und Plexiglas, ist direkt sichtbar, dass Aluminium eine geringere
mittlere Reichweite hat als die anderen Materialien.

Der Grund hierfür ist, dass Aluminium ein Metall ist und stärker auf
Ladungsträger wie Elektronen, aus denen die $\beta$-Strahlung besteht,
reagiert und daher ein größeres Hindernis darstellt. Vergleichbare
Impulsraten gibt es bei einer dicke von 0,15cm Gummi bzw. 770$\mu m$
Aluminium bei einer Rate von ca. 3,7Bq, also eine gleiche Impulsrate
bei etwa der doppelten Schichtdicke für Gummi.

Bei der Absorption von $\gamma$-Strahlung, stimmen die Messungen
mit den Erwartungen überein, nämlich das bei zunehmender Schichtdicke
die Impulsrate exponentiell Abnimmt.
\end{document}
