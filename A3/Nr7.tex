%% LyX 2.1.2 created this file.  For more info, see http://www.lyx.org/.
%% Do not edit unless you really know what you are doing.
\documentclass[english]{article}
\usepackage[T1]{fontenc}
\usepackage[latin9]{inputenc}
\usepackage{graphicx}
\usepackage{babel}
\begin{document}
Ohne einen Absorber hat die Impulsrate f�r den $\beta$- Strahler
bei 500V$\pm$12,5V ungef�hr 11 Zerf�lle pro Sekunde. Nun werden mit
zunehmender Schicktdicke Aluminiumabsober vor das Pr�perat gestellt..
Die Messdaten lassen sich in folgendem Diagramm aufzeigen,

\begin{figure}
\protect\caption{$\beta-$Strahlung gegen Aluminumabsorber}
\includegraphics{C:/Users/hp/Desktop/Praktikum/beta_strahlung_gegen_alu}
\end{figure}


in Realit�t n�hern sich die Werte der Achse f�r die Impulsrate nur
asymptotisch an den Nullpunkt an.

Wird nun extrapoliert, d.h. das Verhalten der Impulsrate wird �ber
den gesicherten Bereich hinaus (approximiert) bestimmt, kann die praktische
Reichweite bestimmt werden. 

Daf�r wird angenommen, dass die Impulsrate um 1,5$\frac{Zerf\ddot{a}lle}{Sekunde}$
sinkt, wenn die Dicke des Aluminumabsorbers um 250$\mu m$ zunimmt. 

Bei dem letzten gemessenen Wert betr�gt die Impulsrate in sehr guter
N�herung 4$\frac{Zerf\ddot{a}lle}{Sekunde}$ bei einer Absorberdicke
von 770$\mu m$. Um nun die Dicke des Absorbers f�r 0$\frac{Zerf\ddot{a}lle}{Sekunde}$
abzusch�tzen werden die Annahmen verwendet, also m�ssten zus�tzliche
650$\mu m$ Aluminumabsorber hinzugef�gt werden.

Somit betr�gt die praktische Reichweite der $\beta-$Strahlung bei
Aluminum 1420$\mu m$.
\end{document}
