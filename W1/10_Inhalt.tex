\section{Einleitung}

Die Versuchsreihe befasst sich ausschließlich mit dem Stirling-Motor.
Als theoretische Grundlagen dazu sind insbesondere Kenntnisse von
Temperatur und Wärmemenge, thermodynamischen Zuständen und Zustandsgleichungen
sowie Kreisprozessen notwendig.


\subsection{Temperatur und Wärmemenge und weitere Thermodynamik}

Im wesentlichen ist der Unterschied zwischen Wärmemenge und Temperatur,
dass die Wärmemenge die Masse und die spezifische Wärme mit einbezieht,
also einen Materialabhängigen Wert, das heißt, dass je nach Beschaffenheit
des Materials zwei Stoffe die gleiche Temperatur und die gleiche Masse
haben, dennoch eine Wärmemenge (Energie) in sich tragen und somit
unterschiedlich viel Arbeit verrichten können.

\begin{equation}
c\cdot m\cdot\triangle T=\triangle Q\label{eq:W=0000E4rmemenge}
\end{equation}


Nach dem 1. Hauptsatz der Thermodynamik lässt sich folgende Gleichung
formulieren,
\begin{equation}
Q=\triangle U-W\label{eq:}
\end{equation}


$\triangle U$ ist die Änderung der inneren Energie eines thermodynamischen
Systems, bei einem idealen Gas also die Änderung der kinetischen Energie
der Gasatome, und W ist die verrichtete mechanische Arbeit.

Diese lässt sich für einen isothermen Prozess wie folgt berechnen,

\begin{equation}
W_{isotherm}=-\intop_{V1}^{V2}\frac{nRT}{V}dV\label{eq:mechanische Arbeit}
\end{equation}


dies ist eine Näherung für ideale Gase, n ist die Anzahl der Mole
des Gases, R ist die allgemeine Gaskonstante, T ist die Temperatur,
die bei isothermen-Prozessen konstant ist.

Der Wirkungsgrad einer Wärmekraftmaschine ist definiert als geleistete
Arbeit durch die zugeführte Wärme,
\begin{equation}
\eta=\frac{|W|}{|Q|}\label{eq:Wirkungsgrad}
\end{equation}


es gibt allerdings keine Maschine die einen Wirkungrad von 100\% erreicht.

Im Gegensatz zum Wirkungsgrad wird die Leistungszahl definiert, diese
gilt für Kältemaschinen bzw. Wärmepumpen, es wird mechanische Arbeit
aufgewendet um eine Wärmemenge zu übertragen, es gilt:
\begin{equation}
\epsilon=\frac{|Q|}{|W|}\label{eq:Leistungszahl}
\end{equation}



\subsection{Stirling-Kreisprozess}

Der idealisierte Stirling-Kreisprozess besteht aus vier Schritten.
Im ersten Schritt wird bei isothermer Ausdehnung mechanische Arbeit
vom System abgegeben, während ihm eine gleich große Wärmemenge zugeführt
wird. Im zweiten Schritt gibt das System durch isochore Dekompression
Energie in Form von Wärme ab. Im dritten Schritt wird das System isotherm
auf das Ausgangsvolumen zurückgebracht. Dabei nimmt es die gleiche
Arbeit auf, wie es an wärme ab gibt, jedoch weniger als in Schritt
2. Der letzte Schritt ist die isochore Erwärmung, dabei wird der Verdrängerkolben
nach oben bewegt und die Luft nimmt Wärmeenergie auf.

Wie für alle reversiblen Kreisprozesse ist auch hier die obere Schranke
des Wirkunsgrades der Carnot-Wirkungsgrad 
\begin{equation}
\eta_{Carnot}=1-\frac{T_{Kalt}}{T_{Warm}}\label{eq:Carnot}
\end{equation}


dieser wird allerdings niemals von realen Kreisprozessen erreicht,
da es sich hier um eine idealisierte Ansicht handelt.

\begin{figure}[h!]
\centering
	\begin{tikzpicture}[]
\begin{axis}[ymax = 7, ymin = 2, xmin = 0.17, xmax = .4, samples=1000,
		axis x line = bottom, axis y line = left,
		ytick = {6,4,5,3,4}, yticklabels = {$p_1$, $p_4$, $p_2$, $p_3$},
		xtick = {.25, .3333}, xticklabels = {$V_1$, $V_2$}]

\addplot [no marks, domain=.2:1] {x^(-1)};
\addplot [no marks, domain=.15:1] {1.5*x^(-1)};
\addplot+ [no marks] coordinates{(.25,4) (.25,6)};
\addplot+ [no marks] coordinates{(1/3,3) (1/3,4.5)};
\end{axis}
\end{tikzpicture}
	\caption{$ (p,V) $-Diagramm des idealen Stirling-Motors. Laufrichtung im Uhrzeigersinn}
\end{figure}

\newpage
\section{Versuchsteil}

Die Versuche befassen sich mit den beiden Betriebsmodi des Stirling-Motors. Es kann entweder mechanische Arbeit aufgebracht werden, um einen Wärmestrom zu erzeugen oder aus einem Temperaturgefälle mechanische Arbeit erzeugt werden. 

\subsection{Bestimmung der Reibunsverluste}
Wie jeder reale Prozess weicht auch der Stirling-Motor vom idealisierten Konzept ab. Dafür sind auch Reibungsverluste durch die Reibung des Kolbens am Zylinder verantwortlich. Um diese zu bestimmen treibt man den Stirling-Motor an und misst die entstehende Reibungswärme. Diese erhält man aus der Flussrate und der Erwärmung des Kühlwassers. \\
Die gemessene Temperatur im Kühlsystem betrug bei Versuchsbeginn $ \SI{22.4(1)}{\degreeCelsius} $ und blieb nach einiger Zeit konstant bei $ \SI{22.7(1)}{\degreeCelsius} $. Die aus mehreren Messungen gemittelte Abflussrate des Kühlwassers beträgt $ \SI{4.597(286)}{\cubic\centi\meter\per\second} $. Aus der Multiplikation von Abflussrate, Temperaturzunahme und Wärmekapazität von Wasser erhält man daraus, das die Reibungsverluste $ P_{\mathrm{reib}} \SI{5.7715(38645)}{\watt} $ betragen. \\
Aus der Fouriertransformation erhalten wir die Drehfrequenz des Motors von $ f = \SI{3.00(5)}{\per\second} $. Die Verlustleistung pro Zyklus beträgt somit $ W_\mathrm{ver} = P_{\mathrm{reib}}\cdot \tau = \frac{P_{\mathrm{reib}}}{f} = \SI{1,9238 \pm 1,2886}{\joule} $.\\
Dass sich dieser Wert nicht genauer bestimmen lässt, ist hauptsächlich eine Folge der ungenauen Temperaturmessung. Die Temperaturdifferenz betrug bei uns gerade mal $ \SI{.3}{\degreeCelsius} $ bei einer Unsicherheit von $ \SI{.2}{\degreeCelsius} $, da sowohl die untere als auch die obere Temperatur nur auf $ \SI{.1}{\degreeCelsius} $ genau bestimmt werden konnten.

\subsection{Bestimmung der Kühlleistung}

Zur Bestimmung der Kühlleistung werden etwa $ \SI{1}{\milli\liter} $ Wasser im Stirling-Motor bis auf $ \SI{-25}{\degreeCelsius} $ abgekühlt. Dazu wird das Wasser im oberen Ende des Stirling-Motors eingesetzt, und im Uhrzeigersinn bei konstanter Drehfrequenz angetrieben. Die Drehfrequenz sollte vom ersten Versuch übernommen werden, also wieder bei etwa $ \SI{3}{\per\second} $ liegen. \\

\begin{figure}[h!]
	\centering
	\begin{tikzpicture}
\begin{axis}[width = .9\textwidth, height = 9cm,
		xmin = 0,
%		axis x line = middle,
		xlabel = {Zeit t $ [\si{\second}] $},
		ylabel = {Wassertemperatur $ [\si{\degreeCelsius}] $}]
	\addplot+ [only marks, mark = o, mark options={scale=.04}] table {diagramme/a2T.data};
\end{axis}
\end{tikzpicture}
	\caption{Temperaturverlauf bei Kühlung durch Stirling-Motor}
	\label{fig:a2T}
\end{figure}

Den Versuch haben wir wie den ersten bei einer Frequenz von $ \SI{3,00(5)}{\per\second} $ durchgeführt. Der Verlauf der Kurve weißt bei etwa $ \SI{0}{\degreeCelsius} $ ein Plateu auf, während die Temperatur links und rechts davon in guter Näherung linear fällt. Erst gegen Ende der Messung flacht die Kurve wieder ab. Zudem unterschreitet die Temperaturkurve bereits vor dem Plateu $ \SI{0}{\degreeCelsius} $, steigt dann aber Sprungartig auf an.\\
Das Plateu lässt sich durch die frei werdende Schmelzwärme beim Übergang $ fl"ussig \rightarrow fest $ erklären. Nach dem Beginn des Gefrierprozesses kann die Temperatur durch die Schmelzwärme nicht weiter sinken. Dennoch wird weiterhin kontinuierlich die gleiche Energiemenge dem Wasser entzogen wie vorher. Daraus lässt sich die Schmelzwärme von Wasser bestimmen:\\
Die durchschnittliche Temperaturänderung von $ t_1 = \SI{60.0(1)}{\second} $ bis $ t_2 = \SI{154,0(1)}{\second} $ lässt sich durch ein Steigungsdreieck bestimmen. Die dazugehörigen Temperaturen sind $ T_1 = \SI{12,2(1)}{\degreeCelsius} $ und $ T_2 = \SI{-0,8(1)}{\degreeCelsius} $. Die Temperaturänderungsrate beträgt somit $ \SI{-0,13830 \pm 0,00215}{\kelvin\per\second} $. Durch lineare Extrapolation ergibt sich daraus für $ t_3 = \SI{480\pm .1}{\second} $ eine Temperatur von $ T_{3,e} = \SI{-45,885 \pm 0,721}{\degreeCelsius} $; die tatsächliche Temperatur beträgt $ T_3 = \SI{-8.2(1)}{\degreeCelsius} $. Das Wasser ist also durch die Schmelzwärme $ \Delta T = \SI{37,685\pm0,821}{\kelvin} $ Wärmer geblieben. Dies entspricht einer Wärmemenge von $ Q_\mathrm{schmelz} = \SI{157,712\pm3,014}{\joule\per\gram} $. \\
Der von uns bestimmte Wert entspricht in etwa der Hälfte des auf Wikipedia.de\cite{wiki:eigenschaftenWasser} angegebenen Referenzwertes von $ \SI{333,5}{\joule\per\gram} $. Wir hatten bei der Messung Probleme damit, das sich die Zeit immer wieder zurückgesetzt hat. Außerdem mussten wir im Nachhinein feststellen, dass das Programm die eingestellte Messfrequenz während der Messung von $ \SI{5}{\per\second} $ auf $ \SI{10}{\per\second} $ geändert hat. Dies könnte auch den Faktor $ 2 $ um den unser Ergebnis abweicht erklären.\\
Um die Kühlleitstung zu bestimmen sollte die Temperatur möglichst nah bei der Umgebungstemperatur liegen, da ansonsten die Wärmeaufnahme von der Umgebung zu großen Einfluss nimmt. Jedoch war bei uns die Steigung im Rahmen der systematischen Fehler bei uns am Anfang so konstant, dass auch wieder die Steigung die zur Berechnung der Schmelzwärme angenommen wurde verwendet werden kann. Eine Temperaturänderungsrate von $ \SI{-0,13830 \pm 0,00215}{\kelvin\per\second} $ entspricht bei \SI{1.00(2)}{\gram} Wasser einer Leistung von \SI{0,57878\pm0,00899}{\watt},

\subsection{Bestimmung der Heizleistung}
Zur Bestimmung der Heizleistung wird nun der Motor als Wärmepumpe
betrieben, dazu wird die Umlaufrichtung umgekehrt. In Abbildung 1,
ist der Temperaturverlauf in Abhängigkeit von der Zeit dargestellt.
Am Anfang der Messung, ist für die ersten Sekunden kein Anstieg zu
sehen, das liegt daran, dass diese Messung unmittelbar durchgeführt
wurde, nachdem der Motor als Kältemaschine benutzt wurde, daher wurden
einige Umdrehungen benötigt um den Aufheizprozess einzuleiten. Danach
ist ein relativer linearer Anstieg der Temperatur zu sehen, ab einer
Temperatur von ca. 0°C ist ein Plateu zu sehen, die Erklärung hierfür
ist, dass das Eis nun am schmelzen ist, dies bedeutet, dass die zugeführte
Wärmeenergie nicht in einer Temperaturerhöhung resultiert, sondern
für den Schmelzvorgang benötigt wird. Ist der Schmelzprozess abgeschlossen
steigt die Temperatur Erwartungsgemäß wieder linear an.

Um nun die spezifische Wärme von Eis zu bestimmen, wird zuerst angenommen,
dass die zugeführte Wärmeenergie pro Zeit konstant ist, da außerdem
die spezifische Wärme von Wasser bekannt ist $\left(4,185Jg^{-1}K^{-1}\right)$,
die Masse des Wassers ist in guter Näherung mit $(1\pm0,1)$g approximiert.
Nachdem der Schmelzprozess beendet ist, was bei ca. 8,2°C bei einer
Zeit von 230s der Fall ist, wird der weitere Verlauf als linear angenommen.
Am Ende der Messung hat das Wasser eine Temperatur von 49,1°C bei
der Zeit von 375s, daraus ergibt sich eine Temperaturerhöhung pro
Zeit von$0,282\frac{K}{s}$. Die gesamte zugeführte Wärmeenergie lässt
sich auch berechnen (die spezifische Wärme von Wasser und die Temperatur
werden als genau angenommen)
\[
(1\pm0,1)g\cdot4,185\frac{J}{gK}\cdot40,9K=(171,16\pm17,11)J
\]


daraus ergibt sich eine Energiezufuhr von$(1,20\pm0,12)\frac{J}{s}$.

Das Eis wird in einer Zeit von 53s um 24K erwärmt, also gilt
\[
(1\pm0,1)g\cdot c_{Eis}\cdot24K=(63,6\pm6,4)J
\]


also für die (abgeschätzte) spezifische Wärme
\[
c_{Eis}=(2,650\pm0,37)\frac{J}{gK}
\]


für den Fehler gilt
\[
\triangle c_{Eis}=\sqrt{\left(\frac{\partial c_{Eis}}{\partial m}\triangle m\right)^{2}+\left(\frac{\partial c_{Eis}}{\partial\triangle T}\triangle\triangle T\right)^{2}+\left(\frac{\partial c_{Eis}}{\partial\triangle Q}\triangle\triangle Q\right)^{2}}
\]


dieser Wert stimmt nicht mit dem Literaturwert von $2,060 \frac{J}{gK}$ 
(Quelle: http://de.wikipedia.org/ \cite{wiki:eigenschaftenWasser}) überein, zum einen kann es daran
liegen, dass es sich hier um eine Abschätzung handelt und einige Annahmen
getroffen wurden die das Ergebnis verfälschen. Außerdem können gewisse
Daten nicht mit 100\%iger Genauigkeit geschätzt werden was zu Fehlern
führen kann!

Die Temperatur des Wasser-Reservoirs ist konstant geblieben, bei 23,1°C
und die Temperatur des Kühlwassers ist von 24,3°C auf 22,9°C abgesunken!

\begin{figure}[h!]
	\centering
	\input{diagramme/a3_T.tex}
	\caption{Temperaturverlauf beim heizen durch Stirling-Motor}
	\label{fig:a3T}
\end{figure}

\subsection{Bestimmung des Wirkungsgrades aus dem $ (p,V) $-Diagramm}
In diesem und dem folgendem Versuch wird der Stirling-Motor nicht mehr durch mechanische Arbeit angetrieben, sondern als Wärmekraftmaschine betrieben. Dazu wurde der Stirling-Motor am oberen Ende durch eine elektrische Heizwedel erwärmt, während er am unteren Ende weiterhin durch Wasser gekühlt wurde. Wegen $ \delta W = p\di V $ ist die abgegebene Arbeit pro Zyklus die eingeschlossene Fläche der Kurve im $ (p,V) $-Diagramm. Teilt man durch die Zeit pro Zyklus bzw. multipliziert man mit der Frequenz erhält man die Leistung. Um den Wirkungsgrad zu erhalten braucht man zudem die eingesetzte Energie. Diese wird durch Messung der Spannung und Stromstärke an der Heizwedel bestimmt. Daraus ergeben sich die folgenden Werte:\\
\begin{table}[h!]
	\centering
	\begin{tabular}{r|r|r|c}
	 & el. Leistung $ [\si{\watt}] $ & mech. Leistung $ [\si{\watt}] $ & Wirkungsgrad \\\hline
	 $ \SI{16}{\volt} $ & $ \num{2250.60(3643)} $ & $ \num{23.362} $ & $ \SI{1.0380(168)}{\percent} $ \\
	 $ \SI{14}{\volt} $ & $ \num{1876.73(3305)} $ & $ \num{16.366} $ & $ \SI{0.8721(154)}{\percent} $ \\
	 $ \SI{12}{\volt} $ & $ \num{1494.30(2941)} $ & $ \num{11.442} $ & $ \SI{0.7657(151)}{\percent} $ \\
	 $ \SI{10}{\volt} $ & $ \num{1079.30(2519)} $ & $ \num{7.475} $ & $ \SI{0.6926(162)}{\percent} $ \\
	 $ \SI{8}{\volt} $ & $ \num{717.50(2057)} $ & $ \num{3.822} $ & $ \SI{0.5327(153)}{\percent} $ 
	\end{tabular}
	\caption{Wirkungsgrad aus $ (p,V) $-Diagramm}
	\label{tab:a4}
\end{table}
Die mechanische Leistung wurde aus numerischer Integration über der $ p $-$ V $-Kurve erhalten. Die elektrische Leistung aus dem Produkt von gemessener Spannung und Stromstärke.\\
Der Wirkungsgrad ist mit maximal etwa $ \SI{1}{\percent} $ sehr gering. Dafür Verantwortlich könnten mehrere Faktoren sein. Zunächst ist der Stirling-Kreisprozess an vielen Stellen idealisiert und praktisch nicht umsetzbar. So ist an den $ (p,V) $-Diagrammen gut zu erkennen, dass insbesondere die isochore (de)kompression nicht erkennbar ist. Dies ist insbesondere eine Folge dessen, dass zu keiner Zeit das Volumen konstant gehalten. Weiterhin hat das System weder die Zeit, um während der Aufwärmphase die Temperatur der Heizwedel aufzunehmen, noch um während der Abkühlphase auf die Temperatur des Kühlwassers abzukühlen. Somit ist das Temperaturgefälle effektiv geringer als im idealisierten Kreislauf, so dass nach \ref{eq:carnot} der maximal mögliche Wirkungsgrad sinkt. \\
Eine weitere Störquelle könnte sein, dass die Heizwedel die Wärme nicht gezielt in den Motor abstrahlt sondern auch in die Umgebung. Dadurch wird elektrische Leistung aufgenommen, die dem Kreisprozess in keiner Weise zugeführt wird.
\begin{figure}[h!]
	\centering
	\input{diagramme/a4_pV16.tex}
	\caption{$ (p,V) $-Diagramm bei $ \SI{16}{\volt} $}
	\label{fig:pV16}
\end{figure}
\begin{figure}[h!]
	\centering
	\input{diagramme/a4_pV14.tex}
	\caption{$ (p,V) $-Diagramm bei $ \SI{14}{\volt} $}
	\label{fig:pV14}
\end{figure}
\begin{figure}[h!]
	\centering
	\input{diagramme/a4_pV12.tex}
	\caption{$ (p,V) $-Diagramm bei $ \SI{12}{\volt} $}
	\label{fig:pV12}
\end{figure}
\begin{figure}[h!]
	\centering
	\begin{tikzpicture}
\begin{axis}[width = .9\textwidth, height = 9cm,
		xmin = 180, xmax = 350,
		xtickmin = 190,
		ymin = 70, ymax = 230,
		ytickmin = 90,
		axis x line = bottom,
		axis y line = left,
		axis x discontinuity=crunch,
		axis y discontinuity=crunch,
		enlargelimits = false,
		xlabel = {Volumen V $ [\si{\cubic\centi\meter}] $},
		ylabel = {Druck p $ [\si{\kilo\pascal}] $}]
	\addplot+ [only marks, mark = o, mark options={scale=.04}] table[x=V,y=p] 
		{diagramme/a4pV10.data};
\end{axis}
\end{tikzpicture}
	\caption{$ (p,V) $-Diagramm bei $ \SI{10}{\volt} $}
	\label{fig:pV10}
\end{figure}
\begin{figure}[h!]
	\centering
	\input{diagramme/a4_pV08.tex}
	\caption{$ (p,V) $-Diagramm bei $ \SI{08}{\volt} $}
	\label{fig:pV08}
\end{figure}
\newpage
\subsection{Bestimmung des Wirkungsgrades durch Abbremsen}

Um den Wirkungsgrad durch Abbremsen zu bestimmen, wird eine Reibungsbremse
benutzt, dem Prony'schen Zaum. Dieser kann an das Schwungrad befestigt
werden und mithilfe von zwei Schrauben je nach nötiger Reibung einstellbar.
Bei Änderung der Reibungskraft, verändert sich auch die Frequenz mit
der das System schwingt.

Das Schwungrad übt ein Drehmoment auf den Zaum aus, dieser wird mithilfe
einer Federwaage gemessen. Das Federpendel lässt allerdings nur Messungen
bis zu 1N zu, daher sind zusätzlich noch Gewichte vorhanden die den
Zaum nochmal mit 0,5N ,,entlasten``, damit Messungen >1N möglich
sind.

Bei dieser Messreihe war es nicht möglich ein Drehmoment von über
0,75N an der Federwaage plus ein Zusatzgewicht zu erreichen (=1,25N),
da der der Stirlingmotor dann so stark gebremst wurde, dass er nicht
mehr schwang. 

Der Wirkungsgrad wird aus dem Quotienten von mechanischer Leistung
geteilt durch die zugeführte elektrische Leistung ermittelt.
\[
\eta=\frac{P_{Zaum}}{P_{elektrisch}}=\frac{2\pi fFr}{UI}
\]


für den Fehler gilt nach Gauß'scher Fehlerfortpflanzung
\[
\triangle\eta=\sqrt{\left(\frac{\partial\eta}{\partial P_{Zaum}}\triangle P_{Zaum}\right)^{2}+\left(\frac{\partial\eta}{\partial P_{elektrisch}}\triangle P_{elektrisch}\right)^{2}}
\]


Die elektrische Leistung beträgt ( wie in Teil 4 ermittelt) 2250,6W.
Der Wirkungsgrad liegt hier in einem Bereich von $10^{-4}$vor, ist
also ziemlich gering.

Der maximale Wirkungsgrad liegt hier bei einer Frequenz von 3,4Hz
vor.

Auch liegt hier eine Schwankung vor die nicht in den allgemeinen Trend
passt, nämlich zwischen 4Hz und 3,5Hz sinkt der Wirkungsgrad wieder
etwas, wahrscheinlich wurde ein Messfehler begangen, indem die Federwaage
oder der Zaum nicht richtig eingestellt waren.

\begin{figure}
\protect\caption{Wirkungsgrad gegen Frequenz}
\end{figure}
