\section{Einleitung}

Die Versuchsreihe befasst sich ausschließlich mit dem Stirling-Motor. Als theoretische Grundlagen dazu sind insbesondere Kenntnisse von Temperatur und Wärmemenge, thermodynamischen Zuständen und Zustandsgleichungen sowie Kreisprozessen notwendig.

\subsection{Temperatur und Wärmemenge} % Und weitere Thermodynamik


\subsection{Stirling-Kreisprozess}
Der idealisierte Stirling-Kreisprozess besteht aus vier Schritten. Im ersten Schritt wird bei isothermer Ausdehnung mechanische Arbeit vom System abgegeben, während ihm eine gleich große Wärmemenge zugeführt wird. Im zweiten Schritt gibt das System durch isochore Dekompression Energie in Form von Wärme ab. Im dritten Schritt wird das System isotherm auf das Ausgangsvolumen zurückgebracht. Dabei nimmt es die gleiche Arbeit auf, wie es an wärme ab gibt, jedoch weniger als in Schritt 2. Der letzte Schritt ist die isochore Kompression. Nach Aufnahme der gleichen Wärmemenge, wie das System im dritten Schritt abgegeben hat, kehrt das System in den Ausgangszustand zurück. Der Stirling-Kreisprozess ist somit reversibel. \\
Wie für alle reversiblen Kreisprozesse ist auch hier die obere Schranke des Wirkunsgrades der Carnot-Wirkungsgrad
\begin{equation}
	\eta \leq \eta_c = 1 - \frac{T_{kalt}}{T_{warm}} \label{eq:carnot}\text{.}
\end{equation}
Für den idealen Stirling-Kreisprozess gilt $ \eta = \eta_c $.

\subsection{Stirling-Motor}

\newpage
\section{Versuchsteil}

Die Versuche befassen sich mit den beiden Betriebsmodi des Stirling-Motors. Es kann entweder mechanische Arbeit aufgebracht werden, um einen Wärmestrom zu erzeugen oder aus einem Temperaturgefälle mechanische Arbeit erzeugt werden. 

\subsection{Bestimmung der Reibunsverluste}
Wie jeder reale Prozess weicht auch der Stirling-Motor vom idealisierten Konzept ab. Dafür sind Reibungsverluste durch die Reibung des Kolbens am Zylinder verantwortlich. Um diese zu bestimmen treibt man den Stirling-Motor an und misst die entstehende Reibungswärme. Diese erhält man aus der Flussrate und der Erwärmung des Kühlwassers. \\
Die gemessene Temperatur im Kühlsystem betrug bei Versuchsbeginn $ \SI{22.4(1)}{\degreeCelsius} $ und blieb nach einiger Zeit konstant bei $ \SI{22.7}{\degreeCelsius} $. Die aus mehreren Messungen gemittelte Abflussrate des Kühlwassers beträgt $ \SI{4.597(286)}{\cubic\centi\meter\per\second} $.

\subsection{Bestimmung der Kühlleistung}

Zur Bestimmung der Kühlleistung werden etwa $ \SI{1}{\milli\liter} $ Wasser im Stirling-Motor bis auf $ \SI{-25}{\degreeCelsius} $ abgekühlt. Dazu wird das Wasser im oberen Ende des Stirling-Motors eingesetzt, und im Uhrzeigersinn bei konstanter Drehfrequenz angetrieben. Die Drehfrequenz sollte vom ersten Versuch übernommen werden, also wieder bei etwa $ \SI{3}{\per\second} $ liegen. 

\begin{figure}[h!]
	\centering
	\begin{tikzpicture}
\begin{axis}[width = .9\textwidth, height = 9cm,
		xmin = 0,
%		axis x line = middle,
		xlabel = {Zeit t $ [\si{\second}] $},
		ylabel = {Wassertemperatur $ [\si{\degreeCelsius}] $}]
	\addplot+ [only marks, mark = o, mark options={scale=.04}] table {diagramme/a2T.data};
\end{axis}
\end{tikzpicture}
	\caption{Temperaturverlauf bei Kühlung durch Stirling-Motor}
	\label{fig:a2T}
\end{figure}

\subsection{Bestimmung der Heizleistung}

\begin{figure}[h!]
	\centering
	\begin{tikzpicture}
\begin{axis}[width = .9\textwidth, height = 9cm,
		xmin = 0,
%		axis x line = middle,
		xlabel = {Zeit t $ [\si{\second}] $},
		ylabel = {Wassertemperatur $ [\si{\degreeCelsius}] $}]
	\addplot+ [only marks, mark = o, mark options={scale=.04}] table {diagramme/a3T.data};
\end{axis}
\end{tikzpicture}
	\caption{Temperaturverlauf beim heizen durch Stirling-Motor}
	\label{fig:a3T}
\end{figure}

\subsection{Bestimmung des Wirkungsgrades aus dem $ (p,V) $-Diagramm}
In diesem und dem folgendem Versuch wird der Stirling-Motor nicht mehr durch mechanische Arbeit angetrieben, sondern als Wärmekraftmaschine betrieben. Dazu wurde der Stirling-Motor am oberen Ende durch eine elektrische Heizwedel erwärmt, während er am unteren Ende weiterhin durch Wasser gekühlt wurde. Wegen $ \delta W = p\di V $ ist die abgegebene Arbeit pro Zyklus die eingeschlossene Fläche der Kurve im $ (p,V) $-Diagramm. Teilt man durch die Zeit pro Zyklus bzw. multipliziert man mit der Frequenz erhält man die Leistung. Um den Wirkungsgrad zu erhalten braucht man zudem die eingesetzte Energie. Diese wird durch Messung der Spannung und Stromstärke an der Heizwedel bestimmt. Daraus ergeben sich die folgenden Werte:\\
\begin{table}[h!]
	\centering
	\begin{tabular}{r|r|r|c}
	 & el. Leistung $ [\si{\watt}] $ & mech. Leistung $ [\si{\watt}] $ & Wirkungsgrad \\\hline
	 $ \SI{16}{\volt} $ & $ \num{2250.60(3643)} $ & $ \num{23.362} $ & $ \SI{1.0380(168)}{\percent} $ \\
	 $ \SI{14}{\volt} $ & $ \num{1876.73(3305)} $ & $ \num{16.366} $ & $ \SI{0.8721(154)}{\percent} $ \\
	 $ \SI{12}{\volt} $ & $ \num{1494.30(2941)} $ & $ \num{11.442} $ & $ \SI{0.7657(151)}{\percent} $ \\
	 $ \SI{10}{\volt} $ & $ \num{1079.30(2519)} $ & $ \num{7.475} $ & $ \SI{0.6926(162)}{\percent} $ \\
	 $ \SI{8}{\volt} $ & $ \num{717.50(2057)} $ & $ \num{3.822} $ & $ \SI{0.5327(153)}{\percent} $ 
	\end{tabular}
	\caption{Wirkungsgrad aus $ (p,V) $-Diagramm}
	\label{tab:a4}
\end{table}
Die mechanische Leistung wurde aus numerischer Integration über der $ p $-$ V $-Kurve erhalten. Die elektrische Leistung aus dem Produkt von gemessener Spannung und Stromstärke.\\
Der Wirkungsgrad ist mit maximal etwa $ \SI{1}{\percent} $ sehr gering. Dafür Verantwortlich könnten mehrere Faktoren sein. Zunächst ist der Stirling-Kreisprozess an vielen Stellen idealisiert und praktisch nicht umsetzbar. So ist an den $ (p,V) $-Diagrammen gut zu erkennen, dass insbesondere die isochore (de)kompression nicht erkennbar ist. Dies ist insbesondere eine Folge dessen, dass zu keiner Zeit das Volumen konstant gehalten. Weiterhin hat das System weder die Zeit, um während der Aufwärmphase die Temperatur der Heizwedel aufzunehmen, noch um während der Abkühlphase auf die Temperatur des Kühlwassers abzukühlen. Somit ist das Temperaturgefälle effektiv geringer als im idealisierten Kreislauf, so dass nach \ref{eq:carnot} der maximal mögliche Wirkungsgrad sinkt. \\
Eine weitere Störquelle könnte sein, dass die Heizwedel die Wärme nicht gezielt in den Motor abstrahlt sondern auch in die Umgebung. Dadurch wird elektrische Leistung aufgenommen, die dem Kreisprozess in keiner Weise zugeführt wird.
\begin{figure}[h!]
	\centering
	\begin{tikzpicture}
\begin{axis}[width = .9\textwidth, height = 9cm,
		xmin = 180, xmax = 350,
		xtickmin = 190,
		ymin = 70, ymax = 230,
		ytickmin = 90,
		axis x line = bottom,
		axis y line = left,
		axis x discontinuity=crunch,
		axis y discontinuity=crunch,
		enlargelimits = false,
		xlabel = {Volumen V $ [\si{\cubic\centi\meter}] $},
		ylabel = {Druck p $ [\si{\kilo\pascal}] $}]
	\addplot+ [only marks, mark = o, mark options={scale=.04}] table[x=V,y=p] 
		{diagramme/a4pV16.data};
\end{axis}
\end{tikzpicture}
	\caption{$ (p,V) $-Diagramm bei $ \SI{16}{\volt} $}
	\label{fig:pV16}
\end{figure}
\begin{figure}[h!]
	\centering
	\begin{tikzpicture}
\begin{axis}[width = .9\textwidth, height = 9cm,
		xmin = 180, xmax = 350,
		xtickmin = 190,
		ymin = 70, ymax = 230,
		ytickmin = 90,
		axis x line = bottom,
		axis y line = left,
		axis x discontinuity=crunch,
		axis y discontinuity=crunch,
		enlargelimits = false,
		xlabel = {Volumen V $ [\si{\cubic\centi\meter}] $},
		ylabel = {Druck p $ [\si{\kilo\pascal}] $}]
	\addplot+ [only marks, mark = o, mark options={scale=.04}] table[x=V,y=p] 
		{diagramme/a4pV16.data};
\end{axis}
\end{tikzpicture}
	\caption{$ (p,V) $-Diagramm bei $ \SI{14}{\volt} $}
	\label{fig:pV14}
\end{figure}
\begin{figure}[h!]
	\centering
	\begin{tikzpicture}
\begin{axis}[width = .9\textwidth, height = 9cm,
		xmin = 180, xmax = 350,
		xtickmin = 190,
		ymin = 70, ymax = 230,
		ytickmin = 90,
		axis x line = bottom,
		axis y line = left,
		axis x discontinuity=crunch,
		axis y discontinuity=crunch,
		enlargelimits = false,
		xlabel = {Volumen V $ [\si{\cubic\centi\meter}] $},
		ylabel = {Druck p $ [\si{\kilo\pascal}] $}]
	\addplot+ [only marks, mark = o, mark options={scale=.04}] table[x=V,y=p] 
		{diagramme/a4pV16.data};
\end{axis}
\end{tikzpicture}
	\caption{$ (p,V) $-Diagramm bei $ \SI{12}{\volt} $}
	\label{fig:pV12}
\end{figure}
\begin{figure}[h!]
	\centering
	\begin{tikzpicture}
\begin{axis}[width = .9\textwidth, height = 9cm,
		xmin = 180, xmax = 350,
		xtickmin = 190,
		ymin = 70, ymax = 230,
		ytickmin = 90,
		axis x line = bottom,
		axis y line = left,
		axis x discontinuity=crunch,
		axis y discontinuity=crunch,
		enlargelimits = false,
		xlabel = {Volumen V $ [\si{\cubic\centi\meter}] $},
		ylabel = {Druck p $ [\si{\kilo\pascal}] $}]
	\addplot+ [only marks, mark = o, mark options={scale=.04}] table[x=V,y=p] 
		{diagramme/a4pV10.data};
\end{axis}
\end{tikzpicture}
	\caption{$ (p,V) $-Diagramm bei $ \SI{10}{\volt} $}
	\label{fig:pV10}
\end{figure}
\begin{figure}[h!]
	\centering
	\begin{tikzpicture}
\begin{axis}[width = .9\textwidth, height = 9cm,
		xmin = 180, xmax = 350,
		xtickmin = 190,
		ymin = 70, ymax = 230,
		ytickmin = 90,
		axis x line = bottom,
		axis y line = left,
		axis x discontinuity=crunch,
		axis y discontinuity=crunch,
		enlargelimits = false,
		xlabel = {Volumen V $ [\si{\cubic\centi\meter}] $},
		ylabel = {Druck p $ [\si{\kilo\pascal}] $}]
	\addplot+ [only marks, mark = o, mark options={scale=.04}] table[x=V,y=p] 
		{diagramme/a4pV16.data};
\end{axis}
\end{tikzpicture}
	\caption{$ (p,V) $-Diagramm bei $ \SI{08}{\volt} $}
	\label{fig:pV08}
\end{figure}

