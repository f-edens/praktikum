\section{Einleitung}

\newpage
\section{Versuchsteil}

Die Versuche befassen sich mit den beiden Betriebsmodi des Stirling-Motors. Es kann entweder mechanische Arbeit aufgebracht werden, um einen Wärmestrom zu erzeugen oder aus einem Temperaturgefälle mechanische Arbeit erzeugt werden. 

\subsection{Bestimmung der Reibunsverluste}
Wie jeder reale Prozess weicht auch der Stirling-Motor vom idealisierten Konzept ab. Dafür sind Reibungsverluste durch die Reibung des Kolbens am Zylinder verantwortlich. Um diese zu bestimmen treibt man den Stirling-Motor an und misst die entstehende Reibungswärme. Diese erhält man aus der Flussrate und der Erwärmung des Kühlwassers. \\
Die gemessene Temperatur im Kühlsystem betrug bei Versuchsbeginn $ \SI{22.4(1)}{\degreeCelsius} $ und blieb nach einiger Zeit konstant bei $ \SI{22.7}{\degreeCelsius} $. Die aus mehreren Messungen gemittelte Abflussrate des Kühlwassers beträgt $ \SI{4.597(286)}{\cubic\centi\meter\per\second} $.

\subsection{Bestimmung der Kühlleistung}

\begin{figure}[h!]
	\centering
	\begin{tikzpicture}
\begin{axis}[width = .9\textwidth, height = 9cm,
		xmin = 0,
%		axis x line = middle,
		xlabel = {Zeit t $ [\si{\second}] $},
		ylabel = {Wassertemperatur $ [\si{\degreeCelsius}] $}]
	\addplot+ [only marks, mark = o, mark options={scale=.04}] table {diagramme/a2T.data};
\end{axis}
\end{tikzpicture}
	\caption{Temperaturverlauf bei Kühlung durch Stirling-Motor}
	\label{fig:a2T}
\end{figure}

\subsection{Bestimmung der Heizleistung}

\begin{figure}[h!]
	\centering
	\begin{tikzpicture}
\begin{axis}[width = .9\textwidth, height = 9cm,
		xmin = 0,
%		axis x line = middle,
		xlabel = {Zeit t $ [\si{\second}] $},
		ylabel = {Wassertemperatur $ [\si{\degreeCelsius}] $}]
	\addplot+ [only marks, mark = o, mark options={scale=.04}] table {diagramme/a3T.data};
\end{axis}
\end{tikzpicture}
	\caption{Temperaturverlauf beim heizen durch Stirling-Motor}
	\label{fig:a3T}
\end{figure}

\subsection{Bestimmung des Wirkungsgrades aus dem $ (p,V) $-Diagramm}

\begin{figure}[h!]
	\centering
	\begin{tikzpicture}
\begin{axis}[width = .9\textwidth, height = 9cm,
		xmin = 180, xmax = 350,
		xtickmin = 190,
		ymin = 70, ymax = 230,
		ytickmin = 90,
		axis x line = bottom,
		axis y line = left,
		axis x discontinuity=crunch,
		axis y discontinuity=crunch,
		enlargelimits = false,
		xlabel = {Volumen V $ [\si{\cubic\centi\meter}] $},
		ylabel = {Druck p $ [\si{\kilo\pascal}] $}]
	\addplot+ [only marks, mark = o, mark options={scale=.04}] table[x=V,y=p] 
		{diagramme/a4pV16.data};
\end{axis}
\end{tikzpicture}
	\caption{$ (p,V) $-Diagramm bei $ \SI{16}{\volt} $}
	\label{fig:pV16}
\end{figure}