\section{Einleitung}
In diesem Versuch geht es darum Wellenphänomene anhand von Mikrowellen zu untersuchen.
Mikrowellen haben eine Wellenlänge von 30 \si{cm} bis 1\si{mm}, somit können Wellenphänomene die sonst mit sichtbaren Licht schwer zu beobachten sind besser beobachtbar.
\subsection{Braggsche Refexion}
Bei der Braggschen Reflexion trifft eine ebene Welle unter dem sogenannten Glanzwinkel \alpha auf ein Gitter aus (in diesem Fall) Metallkugeln. Diese sind auf mehreren Ebenen verteilt, der Abstand wird Netzebenenabstand genannt und mit \textit{d} bezeichnet.
Wenn eine Welle in ein solches Gitter einfällt so wird ein Teil von ihr an der ersten Ebene der Metallkugeln reflektiert, die nicht-reflektierte Strahlung wird an einer der folgenden Ebenen reflektiert. Also reflektiert jede Netzebene eine ebene Welle. Diese Wellen sind phasenverschoben und interferieren miteinander. Konstruktive Interferenz tritt auf wenn gilt,
\begin{equation}
2\textit{d}\sin \alpha=n\lambda
\end{equation}
also wenn der Gangunterschied zweier benachbarter Netzebenen reflektierten Wellen ein ganzzahliges Vielfaches der Wellenlänge beträgt.
\subsection{Totalreflexion und Evaneszente Welle}
Von Totalreflexion wird gesprochen, wenn ein Lichtstrahl aus einem optisch dichterem Medium in ein optisch dünneres Medium (n_{1}>n_{2}) eindringt und der transmittierte Strahl mit einem Winkel von 90° = \vartheta_{2} gebrochen wird, somit entsteht kein transmittierender Strahl.
Aus dem Brechungsgesetz,
\begin{equation}
n_{1}\sin \vartheta_{1}=n_{2}\sin\vartheta_{2}
\end{equation}
folgt mit \vartheta_{2}=90°
\begin{equation}
\sin\vartheta_{T}=\frac{n_{2}}{n_{1}}
\end{equation}
Falls \vartheta_{1}>\vartheta_{T} ist, dann wird 100% der einfallenden Welle Reflektiert.
Allerdings existiert auch im Fall der Totalreflexion eine Welle im Medium 2, diese propagiert allerdings parallel zu Grenzfläche. Diese Welle wird allerdings im Abstand von wenigen Wellenlängen vernachlässigbar klein. Diese Welle wird \textit{evaneszente} Welle genannt.
Wird nun ein drittes Medium mit n_{3}>n_{2} so nah an Medium 1 heran, dass die evaneszente Welle eine im Verhältnis große Amplitude hat, kann diese durch Medium 2 in das Medium 3 propagieren und sich dort fortsetzten. Dies wird \textit{frustrierte Totalreflexion} genannt.



\newpage
\section{Auswertung}
\subsection{Strahldivergenz}
Um die Strahldivergenz des Senders zu bestimmen wird die Intensität der Strahlung in vier Abständen gemessen, außerdem wird bei jedem Abstand die seitliche Entfernung gemessen in welcher die Intensität circa 1\% oder 10\% der Ausgangsintensität beträgt. 
Die Messungen für 1\% werden durchgeführt, falls es nicht möglich ist in den gegebenen Abständen 10\% zu messen.
\begin{tabular}{|c|c|c|}
\hline 
Abstand in cm & Spannung in V & Abstand für 1\%/10\% der Intensität und gemessene Spannung rechts \\ 
\hline 
55 & 7,68 & 0,77 V bei 14,5 cm \\ 
\hline 
60 & 7,91 & 0,79V bei 12,9 cm \\ 
\hline 
65 & 2,63 & 0,26V bei 20,3 cm \\ 
\hline 
70 & 2,07 & 0,020V bei 20,7 cm \\ 
\hline 
\end{tabular} 
diese Messung wurde am Anfang durch geführt. Der Intensitätsregler des Senders war zu circa 50\% aufgedreht. 
Die Messung Messung der Divergenz der anderen Seite wurde mit voll aufgedrehter Intensität gemessen.
\begin{tabular}{|c|c|c|}
\hline 
Abstand vom Sender & Gemessene Spannung & Abstand für 1\%/10\% der Intensität und gemessene Spannung links \\ 
\hline 
55 cm & 13,8V & 0,14 V bei 12,8 cm \\ 
\hline 
60 cm & 9,4 V & 0,95 V bei 14 cm \\ 
\hline 
65 cm & 4,6 V & 0,046 V bei 17,6 cm \\ 
\hline 
70 cm & 3,3 V & 0,032 V bei 19,4 \\ 
\hline 
\end{tabular} 
Die Fehler betragen bei den gemessenen Spannungen \pm0,02\SI{V}, der Fehler der Abstände beträgt \pm0,3\Si{cm}.
Diese Messungen zeigen, dass der Sender die Mikrowellen nicht zu 100\% symmetrisch ausstrahlt.
Um die Divergenz zu bestimmen, werden zwei rechtwinklige Dreiecke mit der Hypotenuse von 70\SI{cm} und den Ankatheten 20,7\SI{cm} und 19,4\SI{cm}. Nach geometrischen Überlegungen beträgt der Öffnungswinkel $ \\Theta $ = 33,3°\pm0,07°.
Allerdings befindet sich die (virtuelle) Strahlungsquelle nicht genau im Zentrum des Senders. Dieser weicht um \approx 0,5\SI{cm}  vom Zentrum ab.


\subsection{Totalreflexion}
Um mit Hilfe des PVC-Halbzylinders eine Totalreflexion zu erzeugen wird der Sender so ausgerichtet, dass dieser  auf die runde Oberfläche strahlt. Diese Strahlen werden gebrochen und propagieren durch den Halbzylinder.
Wenn die Mikrowellen auf die Grenzfläche zwischen PCV-Halbzylinder und Luft treffen (n_{PVC}>n_{Luft}), wird der transmittierende Strahl total reflektiert, wenn der Sender im richtigen Winkel auf den Halbzylinder strahlt.
Wird nun der zweite PVC-Halbzylinder in die nah genug an den Ersten gebracht, so entsteht eine frustrierte Totalreflexion.
Die Stärke dieser frustrierte Totalreflexion nimmt mit dem Abstand der Halbzylinder voneinander exponentiell ab.
(hier Grafik einfügen)



\newpage
\section{Diskussion} 