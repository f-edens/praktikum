\section{Einleitung}
In diesem Versuch wird der Adiabatenexponent $\kappa$ mit Hilfe zweier
Methoden bestimmt.

Essentiell sind hier die Poissonschen Gleichungen, welche Druck, Volumen
und Temperatur bei adiabatischen Zustandsänderungen verknüpfen. Die
Poissonschen Gleichungen lauten,

\[
T\cdot V^{\kappa-1}=const'
\]
\begin{equation}
p\cdot V^{\kappa}=const''\label{eq:Poisson}
\end{equation}
\[
\frac{T^{\kappa}}{p^{\kappa-1}}=const'''
\]


unter Verwendung dieser Gleichungen lassen sich geschlossene Formeln
für $\kappa$ herleiten.

$\kappa$ ist der Quotient der molaren Wärmekapazitäten $c_{m,p}$und
$c_{m,V}$.

$c_{m,p}$ist die Wärmekapazität bei konstantem Druck und $c_{m,V}$
die Wärmekapazität bei konstantem Volumen, für $\kappa$ folgt,
\[
\kappa=\frac{c_{m,p}}{c_{m,V}}=\frac{f+2}{f}
\]


$f$ ist die Zahl der Freiheitsgrade.

Die erste Methode ist die Bestimmung von $\kappa$ nach Rüchard-Flammersfeld.
In diesem Versuchsaufbau befindet sich ein Glasrohr. welches mit einem
Gummistopfen an einer großen Flasche befestigt ist. In die Flasche
wird konstant entweder Luft, Argon oder CO2 hineingepumpt. Das Glasrohr
hat auf etwa halber Höhe einen Schlitz durch den Gas entweichen kann,
das Glasrohr besitzt die Fläche A.

In diesem Rohr befindet sich ein Schwingkörper, der durch das zuströmende
Gas und dem Schlitz zu einer harmonischen Schwingung angeregt wird.

Die Kreifrequenz lässt sich herleiten zu,
\begin{equation}
\omega^{2}=+\frac{\kappa p_{0}A^{2}}{mV_{0}}\label{eq:Frequenz}
\end{equation}


mit Hilfe der Schwingungsdauer $T=\frac{2\pi}{\omega}$ folgt für
$\kappa$,
\begin{equation}
\kappa=\frac{4\pi^{2}mV_{0}}{p_{0}A^{2}T^{2}}\label{eq:adiabatenexponent}
\end{equation}


$V_{0}$und $p_{0}$sind Volumen und Druck des Gases in der Gleichgewichtslage,
$m$ ist die Masse des Schwingkörpers.

Somit kann anhand der Schwingungsdauer des Schwingkörpers der Adiabatenexponent
bestimmt werden.

Die zweite Methode ist die Bestimmung von $\kappa$ nach Clément-Desormes.

Bei dieser Methode ist ein großes Glasgefäß mit Luftgefüllt und mit
einem Flüssigketsbarometer verbunden. Während der Belüftungshahn geschlossen
ist wird der Druck im Gefäß erhöht.

Durch richtiges Timing beim Umdrehen des Hahns kann der Expansionsprozess
als adiabatisch angesehen werden. Durch weitere thermodynamische Überlegungen
und Annahmen und unter der Verwendung der Poissonschen Gleichungen
folgt für den Adiabatenexponent,
\begin{equation}
\kappa=\frac{h_{1}}{h_{1}-h_{3}}\label{eq:adiabatenexponent2}
\end{equation}


$h_{1}$ist die Höhe des Flüssigkeitsbarometers nachdem der Druck
erhöht wurde und $h_{3}$ist die Höhe nachdem der Hahn umgedreht wurde.
Mit Hilfe dieser Formel lässt sich mit diesem Versuchsaufbau der Adiabatenexponent
bestimmen.
\newpage

\section{Auswertung}

\subsection{Bestimmung von $ \kappa $ nach Rüchardt-Flammersfeld}


\begin{figure}
\centering
\begin{tikzpicture}
\begin{axis}[width = .9\textwidth, height=8cm,
				xmin = 0, xmax = 3.5,
				xlabel = {Breite Spaltöffnung $ b $ [\si{\milli\meter}]},
				ylabel = {Schwingfrequenz $ f $ [\si{\per\second}]}]
	
	\addplot+[no marks] {-0.00763313 * x + 1.87894717};
	
	\addplot+ [only marks,
		error bars/.cd, y explicit, x explicit, x dir = both, y dir = both] table[x=x,y=y, x error = xerr, y error = yerr] {
x	xerr	y	yerr
3.00	0.05	1.858670875	0.0131324806
2.50	0.05	1.8566176471	0.0137589501
2.00	0.05	1.8642533937	0.0135781554
1.50	0.05	1.8674806967	0.0134838419
1.00	0.05	1.8695014663	0.0137713996
0.50	0.05	1.8770110833	0.0134617572
	};
	
	\legend{$ {\num{-0.0076}\cdot b + \num{1.8789}} $, Messwerte};
\end{axis}
\end{tikzpicture}m
\caption{$ b $-$ f $-Diagramm mit Luft}
\end{figure}

\begin{figure}
\centering
\begin{tikzpicture}
\begin{axis}[width = .9\textwidth, height=8cm,
				xmin = 0, xmax = 3.5,
				xlabel = {Breite Spaltöffnung $ b $ [\si{\milli\meter}]},
				ylabel = {Schwingfrequenz $ f $ [\si{\per\second}]},
				legend pos = south east]
	
	\addplot+[no marks] {0.02397645 * x + 1.90996119};
	
	\addplot+ [only marks,
		error bars/.cd, y explicit, x explicit, x dir = both, y dir = both] table[x=x,y=y, x error = xerr, y error = yerr] {
x	xerr	y	yerr
0.5	0.05	1.9179004038	0.0128241398
1	0.05	1.9361778415	0.0137399449
1.5	0.05	1.9498607242	0.014290247
2	0.05	1.9589376530	0.014533844
2.5	0.05	1.9673348181	0.0143563879
3	0.05	1.9813084112	0.0145182333
	};
	
	\legend{$ {\num{0.0256}\cdot b + \num{1.9100}} $, Messwerte};
\end{axis}
\end{tikzpicture}
\caption{$ b $-$ f $-Diagramm mit Argon}
\end{figure}

\begin{figure}
\centering
\begin{tikzpicture}
\begin{axis}[width = .9\textwidth, height=8cm,
				xmin = 0, xmax = 3.5,
				xlabel = {Breite Spaltöffnung $ b $ [\si{\milli\meter}]},
				ylabel = {Schwingfrequenz $ f $ [\si{\per\second}]}]
	
	\addplot+[no marks] {0.00759521 * x + 1.81033195};
	
	\addplot+ [only marks,
		error bars/.cd, y explicit, x explicit, x dir = both, y dir = both] table[x=x,y=y, x error = xerr, y error = yerr] {
x	xerr	y	yerr
3	0.05	1.8481608987	0.0135278291
2.5	0.05	1.8117854002	0.0129909056
2	0.05	1.8236016132	0.0129957176
1.5	0.05	1.8305084746	0.0125877388
1	0.05	1.8010435954	0.0123912563
.5	0.05	1.8266413924	0.0127833737
	};
	
	\legend{$ {\num{0.0076}\cdot b + \num{1.8104}} $, Messwerte};
\end{axis}
\end{tikzpicture}
\caption{$ b $-$ f $-Diagramm mit \ch{CO2}}
\end{figure}

\subsection{Bestimmung von $ \kappa $ nach Clément-Desormes}
%todo: Durchführung


\begin{table}
\centering
\sisetup{table-figures-integer = 1, table-figures-uncertainty = 1, table-figures-decimal = 1}
\begin{tabular}{
	S[ table-figures-integer = 2]
	S
	S[table-figures-decimal = 3,table-figures-uncertainty = 3]
	S[table-figures-uncertainty = 2, table-figures-decimal = 2]}
{$ h_1 [\si{\centi\meter}] $} & {$ h_3 [\si{\centi\meter}] $} & {$ \kappa $} & {$ f $} \\\hline\hline 
10,6 \pm 0,1 & 2,9 \pm 0,1 & 1,377 \pm 0,019 & 5,31 \pm 0,27 \\
10,6 \pm 0,1 & 2,7 \pm 0,1 & 1,342 \pm 0,018 & 5,85 \pm 0,31 \\
10,2 \pm 0,1 & 2,6 \pm 0,1 & 1,342 \pm 0,019 & 5,85 \pm 0,32 \\
13,5 \pm 0,1 & 3,6 \pm 0,1 & 1,364 \pm 0,015 & 5,50 \pm 0,22 \\
8,5 \pm 0,1 & 2,3 \pm 0,1 & 1,371 \pm 0,023 & 5,39 \pm 0,34 \\
8,2 \pm 0,1 & 2,3 \pm 0,1 & 1,390 \pm 0,025 & 5,13 \pm 0,33
\end{tabular}
\caption{Messwerte...}
\end{table}

\newpage
\section{Diskussion} 
