%% LyX 2.1.2 created this file.  For more info, see http://www.lyx.org/.
%% Do not edit unless you really know what you are doing.
\documentclass[english]{article}
\usepackage[T1]{fontenc}
\usepackage[utf8]{luainputenc}
\usepackage{babel}
\begin{document}

\section{Rüchard Flammersfeld}

Um den Adiabetenexponent nach Rüchardt-Flammersfeld zu bestimmen,
wird in die große Flasche ein Gas gepumpt. Dadurch wird der Schwingkörper
nach oben gedrückt. Wenn der Schwingkörper über dem Schlitz im Glasrohr
gelangt, entweicht das Gas dort hindurch. Das führt dazu, dass der
Druck abnimmt und der Schwingkörper nach unten fällt. Je nach Art
des Gases fällt dieser mehr oder weniger tief hinunter und wird anschließend
wieder hochgedrückt - eine harmonische Schwingung entsteht.

Nun wird die Zeit gemessen in der etwas 100 Schwingungen gezählt werden,
daraus lässt sich die Frequenz der Schwingung und somit auch der Adiabatenexponent
bestimmen.

Es werden die Gase Argon (einatomig, 3 Freiheitsgrade),Luft(in Näherung
zweiatomig, 5 Freiheitsgrade) und CO2(dreiatomig, gestreckt, 5 Freiheitsgrade)
verwendet. Durch zwei Schellen am Glasrohr lässt sich die Spaltbreite
variieren. 
\end{document}
