%% LyX 2.1.2 created this file.  For more info, see http://www.lyx.org/.
%% Do not edit unless you really know what you are doing.
\documentclass[english]{article}
\usepackage[T1]{fontenc}
\usepackage[utf8]{luainputenc}

\makeatletter

%%%%%%%%%%%%%%%%%%%%%%%%%%%%%% LyX specific LaTeX commands.
%% Because html converters don't know tabularnewline
\providecommand{\tabularnewline}{\\}

\makeatother

\usepackage{babel}
\begin{document}

\section{Durchführung}

Für die Bestimmung von $\kappa$ nach Clément-Desormes wurden mehrere
Messungen durchgeführt, die folgende Tabelle enthält die Messdaten
für die Nulllage, Höhe nach Druckerhöhung und Höhe nach Umdrehung
des Hahns, alle Werte sind in cm und haben einen Fehler von $\pm0,1cm$.

$h_{1}$ist die Höhe nach Druckerhöhung - Nullage und $h_{3}$ist
die Höhe nach Umdrehung des Hahns -Nulllage.

\begin{figure}
\protect\caption{Messwerte }
\begin{tabular}{|c|c|c|}
\hline 
Nulllage & Höhe nach Druckerhöhung & Höhe nach Umdrehung des Hahns\tabularnewline
\hline 
\hline 
39,9 & 50,5 & 42,8\tabularnewline
\hline 
39,8 & 50,4 & 42,5\tabularnewline
\hline 
39,9 & 50,1 & 42,5\tabularnewline
\hline 
39,8 & 53,3 & 43,4\tabularnewline
\hline 
39,9 & 48,4 & 42,2\tabularnewline
\hline 
39,8 & 48,0 & 42,1\tabularnewline
\hline 
\end{tabular}
\end{figure}


Nach 1.4 ist der Mittelwert von $\kappa$ aus dieses Werten 

$\kappa_{Luft}=1,36\pm0,4$


\end{document}
