% ################################################################
% #                                                              #
% # Autor: Michael Epping                                        #
% # E-Mail: michael.epping@uni-muenster.de                       #
% # Version: 1.4                                                 #
% # Datum: Juni 2013                                             #
% # Info: Diese Datei sollte nicht verändert werden.             #
% #    Hier werden die Einstellungen festgelegt und              #
% #    Pakete eingebunden. Alles weitere wird über               #
% #    die Dateien verändert, die mit "0X_" beginnen.            #
% # Copyright: CC0 (macht mit diesen Dateien was ihr wollt)      #
% #    https://creativecommons.org/publicdomain/zero/1.0/deed.de #
% #                                                              #
% ################################################################

% Änderungen 1.2 -> 1.3
% * Bei der Verwendung von texlive2012 gibt es Probleme mit myalphadin.
%   Diese Vorlage für Einträge insLiteraturverzeichnis habe ich durch unsrtdin ersetzt.
% * Da ich jetzt mit TeXlipse arbeite, habe ich ein paar Anpassungen vorgenommen.
%   So ist z.B. der Name der bib-Datei fest vorgegeben, damit auch die Autovervollständigung bei BibTeX-Keys funktioniert.
% * Zusätzliche Kommentare sollten das Arbeiten mit dieser Vorlage erleichtern.
% * Die Vorlage enthält sinnvollen Text und nicht nur nutzlose Platzhalter.

% Änderungen 1.3 -> 1.4
% * Das Paket "ifthenx" gibt es unter Ubuntu 12.04 mit texlive 2009-15 nicht. Als alternative habe ich "xifthen" eingetragen.
% * Tabulatoren habe ich durch Leerzeichen ersetzt. Dadurch bleibt das Layout (Einrücken und Position Kommentare) erhalten, 
%   egal mit welchem Editor man die Dateien öffnet.
% * Ich habe einen Copyright-Vermerk hinzugefügt, nämlich dass es im Prizip keines gibt.
% * Neuer Abschnitt: latexmk
% * Neuer Abschnitt: Verbatim
% * Die Bilddatei "titelseite.jpg" wurde entfernt (wegen Copyright), da die Vorlage ab jetzt öffentlich zugänglich sein soll.
% * "README.txt" im Verzeichnis Bilder wurde erstellt.
% * Literaturangabe der Anleitung zur Optik, Wärmelehre und Atomphysik hinzugefügt. 

% ###############
% # Allgemeines #
% ###############

% Zeilen, die mit einem Prozentzeichen beginnen sind Kommentare. 
% Alle verwendeten Funktionen sind mit solchen Kommentaren versehen, so dass man den Zweck der jeweiligen Funktion nachvollziehen kann.

% ######################################
% # Konfigurieren der Dokumentenklasse #
% ######################################

\documentclass[
    a4paper,                                               % Papierformat
    oneside,                                               % Einseitig
    %twoside,                                              % Zweiseitig
    12pt,                                                  % Schriftgröße
    pagesize=auto,                                         % schreibt die Papiergröße korrekt ins Ausgabedokument
    headsepline,                                           % Linie unter der Kopfzeile
    %draft=true                                            % Markiert zu lange und zu kurze Zeilen
]{scrartcl}
% Es gibt die Dokumenttypen scrartcle, srcbook, scrreprt und scrlettr. Diese gehören zum KOME-Skript und sollten für deutsche Texte benutzt werden.
% Für englische Texte wählt man entsprechend article, book, report und letter.
% Es ist  nicht unbedingt zu empfehlen, bei einem bestehendem Dokument, die documentclass zu ändern.

% ####################
% # Pakete einbinden #
% ####################


% Die Folgenden Pakete sind schon eingebunden (siehe 00_Protokoll.tex):
% \usepackage[utf8x]{inputenc}                             % Legt die Zeichenkodierung fest, z.B UTF8
% \usepackage[T1]{fontenc}                                 % Verwendung der Zeichentabelle T1, für deutschsprachige Dokumente sinnvoll
% \usepackage[ngerman,english]{babel}                      % Silbentrennung nach neuer deutscher und englischer Rechtschreibung
% \usepackage{amsmath}                                     % Mathepaket
% \usepackage{ifthenx}                                     % Wird benötigt um \ifthenelse zu benutzen
% \usepackage[pdftex]{graphicx}                            % Zum flexiblen Einbinden von Grafiken, pdftex ist optional
% \usepackage{units}                                       % Ermöglicht die Nutzung von \unit[Zahl]{Einheit}
% \usepackage{setspace}                                    % Einfaches wechseln zwischen unterschiedlichen Zeilenabständen
% \usepackage[pdfpagelabels]{hyperref}                     % Verlinkt Textstellen im PDF Dokument
% \usepackage[font=small,labelfont=bf,labelsep=endash,format=plain]{caption}
%                                                          % Darstellung für Caption s.u.
% \usepackage{subfig}                                      % Bilder nebeneinander
% \usepackage{wrapfig}                                     % Fließtext um Figure-Umgebung
% \usepackage{cite}                                        % Zusatzfunktionen zum zitieren
% \usepackage{scrpage2}                                    % Wird für Kopf- und Fußzeile benötigt
% \usepackage{array,dcolumn}                               % Beide Pakete werden für die Ausrichtung der Tabellenspalten benötigt

% Pakete erweitern LaTeX um zusätzliche Funktionen. Dies ist eine Satz nützlicher Pakete.
% Weitere sollten in der Datei"`01_EigenePakete.tex"' hinzugefügt werden.
\usepackage[utf8]{inputenc}                                                  % Legt die Zeichenkodierung fest, z.B UTF8
\usepackage[T1]{fontenc}                                                      % Verwendung der Zeichentabelle T1, für deutschsprachige Dokumente sinnvoll
\usepackage[ngerman,english]{babel}                                           % Silbentrennung nach neuer deutscher und englischer Rechtschreibung
\usepackage{amsmath}                                                          % Mathepaket
\usepackage{xifthen}                                                          % Wird benötigt um \ifthenelse zu benutzen
\usepackage[pdftex]{graphicx}                                                 % Zum flexiblen Einbinden von Grafiken, pdftex ist optional
\usepackage{units}                                                            % Ermöglicht die Nutzung von \unit[Zahl]{Einheit}
\usepackage{setspace}                                                         % Einfaches wechseln zwischen unterschiedlichen Zeilenabständen
\usepackage[pdfpagelabels]{hyperref}                                          % Verlinkt Textstellen im PDF Dokument
\usepackage[font=small,labelfont=bf,labelsep=endash,format=plain]{caption}    % Darstellung für Caption s.u.
\usepackage{subfig}                                                           % Bilder nebeneinander
\usepackage{wrapfig}                                                          % Fließtext um Figure-Umgebung
%\usepackage{cite}                                                             % Zusatzfunktionen zum zitieren
\usepackage{scrpage2}                                                         % Wird für Kopf- und Fußzeile benötigt
\usepackage{array,dcolumn}                                                    % Beide Pakete werden für die Ausrichtung der Tabellenspalten benötigt



% weitere Pakete einbinden

% % % % % % % % % % % % % % % % % % % % % %
% % Eigene

\usepackage{csquotes}
\usepackage[backend=biber, style=numeric]{biblatex}
\addbibresource{literatur.bib} %Bibliographie

\usepackage[locale=DE]{siunitx}
\sisetup{separate-uncertainty,
	range-phrase = {\text{ bis }}, 
	range-units = brackets,
}
\usepackage{pgfplots}
%\pgfplotsset{/pgf/number format/use comma}
\SendSettingsToPgf
\pgfplotsset{compat=1.9}
%\usepgfplotslibrary{external} 
%\tikzexternalize



\usepackage{isotope}
\usepackage{float}

% ############################
% # Eigene Befehle einbinden #
% ############################

% Eigene Befehle eignen sich gut um Abkürzungen für lange Befehle zu erstellen. Die Syntax ist folgende:
% \newcommand{neuer Befahl}{ein langer Befehl}
% Das folgende Beispiel fügt ein Bild mit bestimmten vorgegebenen Optionen ein:
\newcommand{\cImage}[1]{
    \begin{figure}[h!]
        \centering
        \includegraphics[width=0.50\textwidth]{#1}
    \end{figure}
}

\newcommand{\di}{\mathrm d}

% #1 ist dabei ein Parameter, den man \cImage übergeben muss. In 10_Titelseite.tex wird dieser Befehl verwendet. Der Parameter ist dort Bilder/titelseite.jpg.
% Benötigt man keine Parameter, dann lässt man [1] weg. Werden zusätzliche Parameter benötigt, dann kann man die Zahl auf maximal 9 erhöhen.


% #########################
% # Variablen importieren #
% #########################

% Der Befehl \newcommand kann auch benutzt werden um Variablen zu definieren:

% Nummer laut Praktikumsheft:
    \newcommand{\varNum}{O3}
% Name laut Praktikumsheft:
    \newcommand{\varName}{Spektrometer}
% Datum der Durchführung:
    \newcommand{\varDate}{8. Juni 2015}
% Autoren des Protokolls:
    \newcommand{\varAutor}{Frederik Edens, Dennis Eckermann}
% Nummer der eigenen Gruppe (z.B. "1mo"):
    \newcommand{\varGruppe}{Gruppe 6mo}
% E-Mail-Adressen der Autoren:
    \newcommand{\varEmail}{f\_eden01@uni-muenster.de\\dennis.eckermann@gmx.de}
% E-Mail-Adresse anzeigen (true/false):
    \newcommand{\varZeigeEmail}{true}
% Literaturverzeichnis anzeigen (true/false):
    \newcommand{\varZeigeLiteraturverzeichnis}{true}
% Stil der Einträge im Literaturverzeichnis
    \newcommand{\varLiteraturLayout}{unsrtdin}

\newboolean{show}

% #########################
% # Beginn des Dokumentes #
% #########################

\begin{document}
\selectlanguage{ngerman}                                   % Schreibsprache Deutsch
\onehalfspacing                                            % 1 1/2 facher Zeilenabstand
\addtokomafont{sectioning}{\rmfamily}                      % Schriftsatz
\numberwithin{equation}{section}                           % Nummerierung der Formeln entsprechend der Section (z.B. 1.1)
\addtokomafont{caption}{\small\linespread{1}\selectfont}   % Ändert Schriftgröße und Zeilenabstand bei captions

% Römische Ziffern als Seitenzahlen für Titelseite bis einschließlich dem Inhaltsverzeichnis
\setcounter{page}{1}
\pagenumbering{roman}

% #######################################
% # Kopf- und Fußzeile konfigurieren    #
% #######################################

\ihead{\textit{\varNum\ - \varName }}                      % Innenseite der Kopfzeile
\chead{}                                                   % Mitte der Kopfzeile
\ohead{\textit{\varAutor}}                                 % Außenseite der Kopfzeile
\ifoot{}                                                   % Innnenseite der Fußzeile
\cfoot{- \textit{\pagemark} -}                             % Mitte der Fußzeile
\ofoot{}                                                   % Aussenseite der Fußzeile

% ###################################
% # Ausrichtung der Tabellenspalten #
% ###################################

\newcolumntype{,}[1]{D{,}{,}{#1}}                          % , in Tabellen untereinander stellen
\newcolumntype{p}{D{p}{\pm}{-1}}                           % +- in Tabellen untereinander stellen

% ########################
% # Titelseite einbinden #
% ########################

\begin{titlepage}
    \vspace*{4cm}
    \begin{center}
        \Huge
        \textbf{\varName}\\
        \vspace{1cm}
        \large
        Protokoll zum Versuch Nummer {\varNum} vom \varDate \\
        \vspace{2,5cm}
        \IfFileExists{Bilder/titelseite.png}{
            \cImage{Bilder/titelseite.png}
        } % Nach \IfFileExists muss eine Leerzeile eingefügt werden

        \vspace{1,5cm}
        \varAutor \\  
        \vspace{1cm}
        \normalsize
        \textit{\varGruppe} \\
        \newboolean{showEmail}
        \setboolean{showEmail}{\varZeigeEmail}
        \ifthenelse{\boolean{showEmail}}{{\varEmail}\\}{}  
    \end{center}
\end{titlepage}


% ################################
% # Inhaltsverzeichnis einbinden #
% ################################
\setcounter{page}{1}
\tableofcontents
\newpage

% Zurücksetzen der Seitenzahlen auf arabische Ziffern
\setcounter{page}{1}
\pagenumbering{arabic}

\pagestyle{scrheadings}                                    % Ab hier mit Kopf- und Fußzeile

% ###################################
% # Den Inhalt der Arbeit einbinden #
% ###################################

\section{Einleitung}

Die Versuchsreihe befasst sich ausschließlich mit dem Stirling-Motor. Als theoretische Grundlagen dazu sind insbesondere Kenntnisse von Temperatur und Wärmemenge, thermodynamischen Zuständen und Zustandsgleichungen sowie Kreisprozessen notwendig.

\subsection{Temperatur und Wärmemenge} % Und weitere Thermodynamik


\subsection{Stirling-Kreisprozess}
Der idealisierte Stirling-Kreisprozess besteht aus vier Schritten. Im ersten Schritt wird bei isothermer Ausdehnung mechanische Arbeit vom System abgegeben, während ihm eine gleich große Wärmemenge zugeführt wird. Im zweiten Schritt gibt das System durch isochore Dekompression Energie in Form von Wärme ab. Im dritten Schritt wird das System isotherm auf das Ausgangsvolumen zurückgebracht. Dabei nimmt es die gleiche Arbeit auf, wie es an wärme ab gibt, jedoch weniger als in Schritt 2. Der letzte Schritt ist die isochore Kompression. Nach Aufnahme der gleichen Wärmemenge, wie das System im dritten Schritt abgegeben hat, kehrt das System in den Ausgangszustand zurück. Der Stirling-Kreisprozess ist somit reversibel. \\
Wie für alle reversiblen Kreisprozesse ist auch hier die obere Schranke des Wirkunsgrades der Carnot-Wirkungsgrad
\begin{equation}
	\eta \leq \eta_c = 1 - \frac{T_{kalt}}{T_{warm}} \label{eq:carnot}\text{.}
\end{equation}
Für den idealen Stirling-Kreisprozess gilt $ \eta = \eta_c $.

\subsection{Stirling-Motor}

\newpage
\section{Versuchsteil}

Die Versuche befassen sich mit den beiden Betriebsmodi des Stirling-Motors. Es kann entweder mechanische Arbeit aufgebracht werden, um einen Wärmestrom zu erzeugen oder aus einem Temperaturgefälle mechanische Arbeit erzeugt werden. 

\subsection{Bestimmung der Reibunsverluste}
Wie jeder reale Prozess weicht auch der Stirling-Motor vom idealisierten Konzept ab. Dafür sind auch Reibungsverluste durch die Reibung des Kolbens am Zylinder verantwortlich. Um diese zu bestimmen treibt man den Stirling-Motor an und misst die entstehende Reibungswärme. Diese erhält man aus der Flussrate und der Erwärmung des Kühlwassers. \\
Die gemessene Temperatur im Kühlsystem betrug bei Versuchsbeginn $ \SI{22.4(1)}{\degreeCelsius} $ und blieb nach einiger Zeit konstant bei $ \SI{22.7(1)}{\degreeCelsius} $. Die aus mehreren Messungen gemittelte Abflussrate des Kühlwassers beträgt $ \SI{4.597(286)}{\cubic\centi\meter\per\second} $. Aus der Multiplikation von Abflussrate, Temperaturzunahme und Wärmekapazität von Wasser erhält man daraus, das die Reibungsverluste $ P_{\mathrm{reib}} \SI{5.7715(38645)}{\watt} $ betragen. \\
Aus der Fouriertransformation erhalten wir die Drehfrequenz des Motors von $ f = \SI{3.00(5)}{\per\second} $. Die Verlustleistung pro Zyklus beträgt somit $ W_\mathrm{ver} = P_{\mathrm{reib}}\cdot \tau = \frac{P_{\mathrm{reib}}}{f} = \SI{1,9238 \pm 1,2886}{\joule} $.\\
Dass sich dieser Wert nicht genauer bestimmen lässt, ist hauptsächlich eine Folge der ungenauen Temperaturmessung. Die Temperaturdifferenz betrug bei uns gerade mal $ \SI{.3}{\degreeCelsius} $ bei einer Unsicherheit von $ \SI{.2}{\degreeCelsius} $, da sowohl die untere als auch die obere Temperatur nur auf $ \SI{.1}{\degreeCelsius} $ genau bestimmt werden konnten.

\subsection{Bestimmung der Kühlleistung}

Zur Bestimmung der Kühlleistung werden etwa $ \SI{1}{\milli\liter} $ Wasser im Stirling-Motor bis auf $ \SI{-25}{\degreeCelsius} $ abgekühlt. Dazu wird das Wasser im oberen Ende des Stirling-Motors eingesetzt, und im Uhrzeigersinn bei konstanter Drehfrequenz angetrieben. Die Drehfrequenz sollte vom ersten Versuch übernommen werden, also wieder bei etwa $ \SI{3}{\per\second} $ liegen. \\

\begin{figure}[h!]
	\centering
	\begin{tikzpicture}
\begin{axis}[width = .9\textwidth, height = 9cm,
		xmin = 0,
%		axis x line = middle,
		xlabel = {Zeit t $ [\si{\second}] $},
		ylabel = {Wassertemperatur $ [\si{\degreeCelsius}] $}]
	\addplot+ [only marks, mark = o, mark options={scale=.04}] table {diagramme/a2T.data};
\end{axis}
\end{tikzpicture}
	\caption{Temperaturverlauf bei Kühlung durch Stirling-Motor}
	\label{fig:a2T}
\end{figure}

Den Versuch haben wir wie den ersten bei einer Frequenz von $ \SI{3,00(5)}{\per\second} $ durchgeführt. Der Verlauf der Kurve weißt bei etwa $ \SI{0}{\degreeCelsius} $ ein Plateu auf, während die Temperatur links und rechts davon in guter Näherung linear fällt. Erst gegen Ende der Messung flacht die Kurve wieder ab. Zudem unterschreitet die Temperaturkurve bereits vor dem Plateu $ \SI{0}{\degreeCelsius} $, steigt dann aber Sprungartig auf an.\\
Das Plateu lässt sich durch die frei werdende Schmelzwärme beim Übergang $ fl"ussig \rightarrow fest $ erklären. Nach dem Beginn des Gefrierprozesses kann die Temperatur durch die Schmelzwärme nicht weiter sinken. Dennoch wird weiterhin kontinuierlich die gleiche Energiemenge dem Wasser entzogen wie vorher. Daraus lässt sich die Schmelzwärme von Wasser bestimmen:\\
Die durchschnittliche Temperaturänderung von $ t_1 = \SI{60.0(1)}{\second} $ bis $ t_2 = \SI{154,0(1)}{\second} $ lässt sich durch ein Steigungsdreieck bestimmen. Die dazugehörigen Temperaturen sind $ T_1 = \SI{12,2(1)}{\degreeCelsius} $ und $ T_2 = \SI{-0,8(1)}{\degreeCelsius} $. Die Temperaturänderungsrate beträgt somit $ \SI{-0,13830 \pm 0,00215}{\kelvin\per\second} $. Durch lineare Extrapolation ergibt sich daraus für $ t_3 = \SI{480\pm .1}{\second} $ eine Temperatur von $ T_{3,e} = \SI{-45,885 \pm 0,721}{\degreeCelsius} $; die tatsächliche Temperatur beträgt $ T_3 = \SI{-8.2(1)}{\degreeCelsius} $. Das Wasser ist also durch die Schmelzwärme $ \Delta T = \SI{37,685\pm0,821}{\kelvin} $ Wärmer geblieben. Dies entspricht einer Wärmemenge von $ Q_\mathrm{schmelz} = \SI{157,712\pm3,014}{\joule\per\gram} $. \\
Der von uns bestimmte Wert entspricht in etwa der Hälfte des auf Wikipedia.de\cite{wiki:eigenschaftenWasser} angegebenen Referenzwertes von $ \SI{333,5}{\joule\per\gram} $. Wir hatten bei der Messung Probleme damit, das sich die Zeit immer wieder zurückgesetzt hat. Außerdem mussten wir im Nachhinein feststellen, dass das Programm die eingestellte Messfrequenz während der Messung von $ \SI{5}{\per\second} $ auf $ \SI{10}{\per\second} $ geändert hat. Dies könnte auch den Faktor $ 2 $ um den unser Ergebnis abweicht erklären.\\
Um die Kühlleitstung zu bestimmen sollte die Temperatur möglichst nah bei der Umgebungstemperatur liegen, da ansonsten die Wärmeaufnahme von der Umgebung zu großen Einfluss nimmt. Jedoch war bei uns die Steigung im Rahmen der systematischen Fehler bei uns am Anfang so konstant, dass auch wieder die Steigung die zur Berechnung der Schmelzwärme angenommen wurde verwendet werden kann. Eine Temperaturänderungsrate von $ \SI{-0,13830 \pm 0,00215}{\kelvin\per\second} $ entspricht bei \SI{1.00(2)}{\gram} Wasser einer Leistung von \SI{0,57878\pm0,00899}{\watt},

\subsection{Bestimmung der Heizleistung}

\begin{figure}[h!]
	\centering
	\begin{tikzpicture}
\begin{axis}[width = .9\textwidth, height = 9cm,
		xmin = 0,
%		axis x line = middle,
		xlabel = {Zeit t $ [\si{\second}] $},
		ylabel = {Wassertemperatur $ [\si{\degreeCelsius}] $}]
	\addplot+ [only marks, mark = o, mark options={scale=.04}] table {diagramme/a3T.data};
\end{axis}
\end{tikzpicture}
	\caption{Temperaturverlauf beim heizen durch Stirling-Motor}
	\label{fig:a3T}
\end{figure}

\subsection{Bestimmung des Wirkungsgrades aus dem $ (p,V) $-Diagramm}
In diesem und dem folgendem Versuch wird der Stirling-Motor nicht mehr durch mechanische Arbeit angetrieben, sondern als Wärmekraftmaschine betrieben. Dazu wurde der Stirling-Motor am oberen Ende durch eine elektrische Heizwedel erwärmt, während er am unteren Ende weiterhin durch Wasser gekühlt wurde. Wegen $ \delta W = p\di V $ ist die abgegebene Arbeit pro Zyklus die eingeschlossene Fläche der Kurve im $ (p,V) $-Diagramm. Teilt man durch die Zeit pro Zyklus bzw. multipliziert man mit der Frequenz erhält man die Leistung. Um den Wirkungsgrad zu erhalten braucht man zudem die eingesetzte Energie. Diese wird durch Messung der Spannung und Stromstärke an der Heizwedel bestimmt. Daraus ergeben sich die folgenden Werte:\\
\begin{table}[h!]
	\centering
	\begin{tabular}{r|r|r|c}
	 & el. Leistung $ [\si{\watt}] $ & mech. Leistung $ [\si{\watt}] $ & Wirkungsgrad \\\hline
	 $ \SI{16}{\volt} $ & $ \num{2250.60(3643)} $ & $ \num{23.362} $ & $ \SI{1.0380(168)}{\percent} $ \\
	 $ \SI{14}{\volt} $ & $ \num{1876.73(3305)} $ & $ \num{16.366} $ & $ \SI{0.8721(154)}{\percent} $ \\
	 $ \SI{12}{\volt} $ & $ \num{1494.30(2941)} $ & $ \num{11.442} $ & $ \SI{0.7657(151)}{\percent} $ \\
	 $ \SI{10}{\volt} $ & $ \num{1079.30(2519)} $ & $ \num{7.475} $ & $ \SI{0.6926(162)}{\percent} $ \\
	 $ \SI{8}{\volt} $ & $ \num{717.50(2057)} $ & $ \num{3.822} $ & $ \SI{0.5327(153)}{\percent} $ 
	\end{tabular}
	\caption{Wirkungsgrad aus $ (p,V) $-Diagramm}
	\label{tab:a4}
\end{table}
Die mechanische Leistung wurde aus numerischer Integration über der $ p $-$ V $-Kurve erhalten. Die elektrische Leistung aus dem Produkt von gemessener Spannung und Stromstärke.\\
Der Wirkungsgrad ist mit maximal etwa $ \SI{1}{\percent} $ sehr gering. Dafür Verantwortlich könnten mehrere Faktoren sein. Zunächst ist der Stirling-Kreisprozess an vielen Stellen idealisiert und praktisch nicht umsetzbar. So ist an den $ (p,V) $-Diagrammen gut zu erkennen, dass insbesondere die isochore (de)kompression nicht erkennbar ist. Dies ist insbesondere eine Folge dessen, dass zu keiner Zeit das Volumen konstant gehalten. Weiterhin hat das System weder die Zeit, um während der Aufwärmphase die Temperatur der Heizwedel aufzunehmen, noch um während der Abkühlphase auf die Temperatur des Kühlwassers abzukühlen. Somit ist das Temperaturgefälle effektiv geringer als im idealisierten Kreislauf, so dass nach \ref{eq:carnot} der maximal mögliche Wirkungsgrad sinkt. \\
Eine weitere Störquelle könnte sein, dass die Heizwedel die Wärme nicht gezielt in den Motor abstrahlt sondern auch in die Umgebung. Dadurch wird elektrische Leistung aufgenommen, die dem Kreisprozess in keiner Weise zugeführt wird.
\begin{figure}[h!]
	\centering
	\begin{tikzpicture}
\begin{axis}[width = .9\textwidth, height = 9cm,
		xmin = 180, xmax = 350,
		xtickmin = 190,
		ymin = 70, ymax = 230,
		ytickmin = 90,
		axis x line = bottom,
		axis y line = left,
		axis x discontinuity=crunch,
		axis y discontinuity=crunch,
		enlargelimits = false,
		xlabel = {Volumen V $ [\si{\cubic\centi\meter}] $},
		ylabel = {Druck p $ [\si{\kilo\pascal}] $}]
	\addplot+ [only marks, mark = o, mark options={scale=.04}] table[x=V,y=p] 
		{diagramme/a4pV16.data};
\end{axis}
\end{tikzpicture}
	\caption{$ (p,V) $-Diagramm bei $ \SI{16}{\volt} $}
	\label{fig:pV16}
\end{figure}
\begin{figure}[h!]
	\centering
	\begin{tikzpicture}
\begin{axis}[width = .9\textwidth, height = 9cm,
		xmin = 180, xmax = 350,
		xtickmin = 190,
		ymin = 70, ymax = 230,
		ytickmin = 90,
		axis x line = bottom,
		axis y line = left,
		axis x discontinuity=crunch,
		axis y discontinuity=crunch,
		enlargelimits = false,
		xlabel = {Volumen V $ [\si{\cubic\centi\meter}] $},
		ylabel = {Druck p $ [\si{\kilo\pascal}] $}]
	\addplot+ [only marks, mark = o, mark options={scale=.04}] table[x=V,y=p] 
		{diagramme/a4pV16.data};
\end{axis}
\end{tikzpicture}
	\caption{$ (p,V) $-Diagramm bei $ \SI{14}{\volt} $}
	\label{fig:pV14}
\end{figure}
\begin{figure}[h!]
	\centering
	\begin{tikzpicture}
\begin{axis}[width = .9\textwidth, height = 9cm,
		xmin = 180, xmax = 350,
		xtickmin = 190,
		ymin = 70, ymax = 230,
		ytickmin = 90,
		axis x line = bottom,
		axis y line = left,
		axis x discontinuity=crunch,
		axis y discontinuity=crunch,
		enlargelimits = false,
		xlabel = {Volumen V $ [\si{\cubic\centi\meter}] $},
		ylabel = {Druck p $ [\si{\kilo\pascal}] $}]
	\addplot+ [only marks, mark = o, mark options={scale=.04}] table[x=V,y=p] 
		{diagramme/a4pV16.data};
\end{axis}
\end{tikzpicture}
	\caption{$ (p,V) $-Diagramm bei $ \SI{12}{\volt} $}
	\label{fig:pV12}
\end{figure}
\begin{figure}[h!]
	\centering
	\begin{tikzpicture}
\begin{axis}[width = .9\textwidth, height = 9cm,
		xmin = 180, xmax = 350,
		xtickmin = 190,
		ymin = 70, ymax = 230,
		ytickmin = 90,
		axis x line = bottom,
		axis y line = left,
		axis x discontinuity=crunch,
		axis y discontinuity=crunch,
		enlargelimits = false,
		xlabel = {Volumen V $ [\si{\cubic\centi\meter}] $},
		ylabel = {Druck p $ [\si{\kilo\pascal}] $}]
	\addplot+ [only marks, mark = o, mark options={scale=.04}] table[x=V,y=p] 
		{diagramme/a4pV10.data};
\end{axis}
\end{tikzpicture}
	\caption{$ (p,V) $-Diagramm bei $ \SI{10}{\volt} $}
	\label{fig:pV10}
\end{figure}
\begin{figure}[h!]
	\centering
	\begin{tikzpicture}
\begin{axis}[width = .9\textwidth, height = 9cm,
		xmin = 180, xmax = 350,
		xtickmin = 190,
		ymin = 70, ymax = 230,
		ytickmin = 90,
		axis x line = bottom,
		axis y line = left,
		axis x discontinuity=crunch,
		axis y discontinuity=crunch,
		enlargelimits = false,
		xlabel = {Volumen V $ [\si{\cubic\centi\meter}] $},
		ylabel = {Druck p $ [\si{\kilo\pascal}] $}]
	\addplot+ [only marks, mark = o, mark options={scale=.04}] table[x=V,y=p] 
		{diagramme/a4pV16.data};
\end{axis}
\end{tikzpicture}
	\caption{$ (p,V) $-Diagramm bei $ \SI{08}{\volt} $}
	\label{fig:pV08}
\end{figure}



% ####################
% # Anhang einbinden #
% ####################

% Löscht man die Datei "`20_Anhang.tex"', dann wird kein Anhang erzeugt.
\IfFileExists{20_Anhang}{
    \newpage
    \appendix
    \section{Anhang}
    \label{anhang}

\subsection{Fehlerrechnung}
Für linear errechnete Größen $ y(x_1, \dots, x_n) $ ist der Fehler allgemein gegeben durch
\begin{equation}
	\Delta y = \sqrt{\sum_{i=1}^{n}\left(y\frac{\Delta x_i}{x_i}\right)^2} 
		= |y|\sqrt{\sum_{i=1}^{n}\left(\frac{\Delta x_i}{x_i}\right)^2}  \label{eq:err}
\end{equation}
%\subsubsection{Brechungsindex aus Brewsterwinkel}
%Bei Rechnung mit Winkeln in Bogenmaß ist der Fehler aus der gaußschen Fehlerfortpflanzung gegeben durch
%\begin{align}
%	\Delta n = \left|\frac{\partial n}{\partial \alpha_B}\Delta \alpha_B\right| 
%	=  \left|\frac{\Delta\alpha_B}{\cos^2\alpha_B}\right| \label{eq:err:brew}
%\end{align}


%Links zu den Dokumentationen der verwendeten Pakete.
%
%\begin{itemize}
%    \item inputenc: \url{http://ctan.org/pkg/inputenc}
%    \item fontenc: \url{http://ctan.org/pkg/fontenc}
%    \item babel: \url{http://ctan.org/pkg/babel}
%    \item amsmath: \url{http://ctan.org/pkg/amsmath}
%    \item ifthenx: \url{http://ctan.org/pkg/ifthenx}
%    \item graphicx: \url{http://ctan.org/pkg/graphicx}
%    \item units: \url{http://ctan.org/pkg/units}
%    \item setspace: \url{http://ctan.org/pkg/setspace}
%    \item hyperref: \url{http://ctan.org/pkg/hyperref}
%    \item caption: \url{http://ctan.org/pkg/caption}
%    \item subfig: \url{http://ctan.org/pkg/subfig}
%    \item wrapfig: \url{http://ctan.org/pkg/wrapfig}
%    \item cite: \url{http://ctan.org/pkg/cite}
%    \item scrpage2: \url{http://www.komascript.de/komascriptbestandteile}
%    \item array: \url{http://ctan.org/pkg/array}
%    \item dcolumn: \url{http://ctan.org/pkg/dcolumn}
%\end{itemize}
%
%Leider wird nicht für jedes Pakete eine Dokumentation angeboten.

} % Nach \IfFileExists muss eine Leerzeile eingefügt werden

% ###################################
% # Literaturverzeichnis mit BibTeX #
% ###################################

\setboolean{show}{\varZeigeLiteraturverzeichnis}
\ifthenelse{\boolean{show}}{
    \newpage
    \printbibliography 
  %  \bibliographystyle{\varLiteraturLayout}
}{}

% #######################
% # Ende des Dokumentes #
% #######################

\end{document}
