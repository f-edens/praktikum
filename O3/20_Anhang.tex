\label{anhang}

\subsection{Fehlerrechnung}
Fehler von rein proportional oder antiproportionalen Zusammenhängen lassen sich durch die Fortpflanzung der relativen Fehler berechnen.
\begin{equation}
	\Delta y = \sqrt{\sum_{i=1}^{n}\left(y\frac{\Delta x_i}{x_i}\right)^2} 
		= |y|\sqrt{\sum_{i=1}^{n}\left(\frac{\Delta x_i}{x_i}\right)^2}  \label{eq:err}
\end{equation}

\subsubsection{Kalibrierung des Spektrometers}
Bei der Kalibrierung des Spektrometers war es notwendig, den Sinus des Beugungswinkel zu bestimmen. Nach gaußscher Fehlerfortpflanzung ist der Fehler hierfür gegeben durch
\begin{equation}
	\Delta(\sin\vartheta) = \left|\cos\vartheta\cdot\Delta \vartheta\right| \label{eq:err:sin}
\end{equation}


%Links zu den Dokumentationen der verwendeten Pakete.
%
%\begin{itemize}
%    \item inputenc: \url{http://ctan.org/pkg/inputenc}
%    \item fontenc: \url{http://ctan.org/pkg/fontenc}
%    \item babel: \url{http://ctan.org/pkg/babel}
%    \item amsmath: \url{http://ctan.org/pkg/amsmath}
%    \item ifthenx: \url{http://ctan.org/pkg/ifthenx}
%    \item graphicx: \url{http://ctan.org/pkg/graphicx}
%    \item units: \url{http://ctan.org/pkg/units}
%    \item setspace: \url{http://ctan.org/pkg/setspace}
%    \item hyperref: \url{http://ctan.org/pkg/hyperref}
%    \item caption: \url{http://ctan.org/pkg/caption}
%    \item subfig: \url{http://ctan.org/pkg/subfig}
%    \item wrapfig: \url{http://ctan.org/pkg/wrapfig}
%    \item cite: \url{http://ctan.org/pkg/cite}
%    \item scrpage2: \url{http://www.komascript.de/komascriptbestandteile}
%    \item array: \url{http://ctan.org/pkg/array}
%    \item dcolumn: \url{http://ctan.org/pkg/dcolumn}
%\end{itemize}
%
%Leider wird nicht für jedes Pakete eine Dokumentation angeboten.
